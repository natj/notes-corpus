%\documentclass[referee]{aa}
\documentclass[usenatbib,twocolumn]{aastex63}

\pdfoutput=1
\pdfminorversion=5


\usepackage[utf8]{inputenc}
\usepackage{tabularx}
\usepackage{multirow}
\usepackage{url}
\usepackage{amsmath}	% Advanced maths commands
\usepackage{amssymb}	% Extra maths symbols
\usepackage{graphicx}
\usepackage{color}
\usepackage{natbib}
\usepackage{hyperref}

%\usepackage{newtxtext,newtxmath}
\usepackage{times,txfonts}
\usepackage{bm}

% Depending on your LaTeX fonts installation, you might get better results with one of these:
%\usepackage{mathptmx}
%\usepackage{txfonts}

% Use vector fonts, so it zooms properly in on-screen viewing software
% Don't change these lines unless you know what you are doing
\usepackage[T1]{fontenc}
\usepackage{ae,aecompl}


%landscape figures
%\usepackage{wrapfig}
%\usepackage{lscape}
%\usepackage{rotating}

\bibpunct{(}{)}{;}{a}{}{,} % to follow the A&A style

%Debug addition for collaborators
%\usepackage[switch, modulo]{lineno}
%\linenumbers
%\renewcommand\linenumberfont{\color{red}\normalfont\tiny\sffamily}
%\renewcommand\linenumberfont{\normalfont\tiny\sffamily}

\DeclareUnicodeCharacter{00A0}{ }

\usepackage{listingsutf8}

% configure listings package
\lstset{
         basicstyle=\footnotesize\ttfamily, 
         %numbers=left,               
         numberstyle=\tiny,          
         %stepnumber=2,             
         numbersep=5pt,             
         tabsize=2,                  
         extendedchars=true,         
         breaklines=true,            
         keywordstyle=\color{red},
         frame=b,
 %        keywordstyle=[1]\textbf,   
 %        keywordstyle=[2]\textbf,   
 %        keywordstyle=[3]\textbf,   
 %        keywordstyle=[4]\textbf,   
         %stringstyle=\color{white}\ttfamily, % Farbe der String
         showspaces=false,           
         showtabs=false,            
         xleftmargin=17pt,
         framexleftmargin=17pt,
         framexrightmargin=5pt,
         framexbottommargin=4pt,
         %backgroundcolor=\color{lightgray},
         showstringspaces=false    
}
\lstloadlanguages{C++}
\newcommand{\codeil}[1]{\lstinline!#1!}

\usepackage{amsmath}
\usepackage{amssymb}


%--------------------------------------------------
%Basic commands
\newcommand{\be}{\begin{equation}}
\newcommand{\ee}{\end{equation}}
\newcommand{\bea}{\begin{align}\begin{split}}
\newcommand{\eea}{\end{split}\end{align}}

\newcommand{\fig}[1]{Fig.~\ref{#1}}
\newcommand{\eq}[1]{~(\ref{#1})}
\newcommand{\sect}[1]{Sect.~\ref{#1}}
\newcommand{\appnd}[1]{Appendix \ref{#1}}

\newcommand{\Ten}[2]{\ensuremath{#1 \times 10^{#2}} }


%-------------------------------------------------- 
% Units with proper unit spacing
\newcommand{\unitspace}{\,}

\newcommand{\km}{\ensuremath{\unitspace \mathrm{km}}}
\newcommand{\cm}{\ensuremath{\unitspace \mathrm{cm}}}
\newcommand{\g}{\ensuremath{\unitspace \mathrm{g}}}
\newcommand{\Hz}{\ensuremath{\unitspace \mathrm{Hz}}}
\newcommand{\gcm}{\ensuremath{\unitspace \mathrm{g}\,\mathrm{cm}^{-3}}}
\newcommand{\cms}{\ensuremath{\unitspace \mathrm{cm}\,\mathrm{s}^{-1}}}
\newcommand{\cmss}{\ensuremath{\unitspace \mathrm{cm}\,\mathrm{s}^{-2}}}
\newcommand{\dyncm}{\ensuremath{\unitspace \mathrm{dyn}\,\mathrm{cm}^{-2}}}
\newcommand{\erg}{\ensuremath{\unitspace \mathrm{erg}}}
\newcommand{\ergs}{\ensuremath{\unitspace \mathrm{erg}\,\mathrm{s}^{-1}}}

\newcommand{\Gauss}{\ensuremath{\unitspace \mathrm{G}}}
\newcommand{\Kelvin}{\ensuremath{\unitspace \mathrm{K}}}

%\newcommand{\Msun}{\ensuremath{\unitspace M_{\odot}}}
\newcommand{\Msun}{\ensuremath{\unitspace \mathrm{M}_{\odot}}}


%--------------------------------------------------
% colors
\newcommand{\red}[1]{\textcolor{red}{#1}}
\newcommand{\green}[1]{\textcolor{green}{#1}}
\newcommand{\blue}[1]{\textcolor{blue}{#1}}



%--------------------------------------------------
% vectors

%normal 3-vectors
%\renewcommand{\vec}[1]{\ensuremath{\mathbf{#1}}}
%\renewcommand{\vec}[1]{\ensuremath{\boldsymbol{#1}}}
\renewcommand{\vec}[1]{\ensuremath{\bm{\mathit{#1}}}}

%four-vectors
\makeatletter
\def\fvec#1{\underline{\sbox\tw@{$#1$}\dp\tw@\z@\box\tw@}}
\makeatother


%--------------------------------------------------
% math

% integrals
\newcommand{\ud}{\mathrm{d}} % differential d

% derivatives
\newcommand{\pD}[2]{\ensuremath{\partial_{#1}}} %partial derivative
\newcommand{\fracpD}[2]{\ensuremath{\frac{\partial {#1} }{\partial {#2} }}} %frac partial

\newcommand{\derfrac}[2]{\frac{\ud #1}{\ud #2}} % fractional deriv
\newcommand{\parfrac}[2]{\frac{\partial #1}{\partial #2}} %fractional partial deriv
\newcommand{\parD}[1]{\partial_{#1}} %partial derivative shorthand

%misc
\newcommand{\osim}{\mathord{\sim}} % ordo O

% statistics
\newcommand{\bra}[1]{\ensuremath{\langle #1 \rangle}}
\newcommand{\braket}[2]{\ensuremath{ \langle #1 ~|~ #2 \rangle}}


%--------------------------------------------------
% programming

% writing c++ is highly non-trivial
%\newcommand{\cpp}{C\nolinebreak\hspace{-.05em}\raisebox{.4ex}{\tiny\bf +}\nolinebreak\hspace{-.10em}\raisebox{.4ex}{\tiny\bf +{rbrace}{rbrace}}}
%\newcommand{\cpp}{C\nolinebreak\hspace{-.05em}\raisebox{.4ex}{\tiny\bf +}\nolinebreak\hspace{-.10em}\raisebox{.4ex}{\tiny\bf +}}
%\def\cpp{{C\nolinebreak[4]\hspace{-.05em}\raisebox{.4ex}{\tiny\bf ++}}}
%\def\CC{lbrace}{lbrace}C\nolinebreak[4]\hspace{-.05em}\raisebox{.4ex}{\tiny\bf ++{rbrace}{rbrace}{rbrace}
%\usepackage{relsize}
%\newcommand\cpp{C\nolinebreak\hspace{-.05em}\raisebox{.4ex}{\relsize{-3}{\textbf{+}}}\nolinebreak\hspace{-.10em}\raisebox{.4ex}{\relsize{-3}{\textbf{+}}}}
%\newcommand\cpp{C\nolinebreak[4]\hspace{-.05em}\raisebox{.4ex}{\relsize{-3}{\textbf{++}}}}

\newcommand{\cpp}{C\texttt{++}}
\newcommand{\cppv}[1]{C\texttt{++}$#1$}

\newcommand{\python}{\textsc{Python}}





\newcommand{\Cfl}{\hat{c}}
\newcommand{\hf}{\frac{1}{2}}

\newcommand{\Bhat}{\vec{\hat{B}}}
\newcommand{\Ehat}{\vec{\hat{E}}}
\newcommand{\Jhat}{\vec{\hat{J}}}
\newcommand{\uhat}{\vec{\hat{u}}}
\newcommand{\vhat}{\vec{\hat{v}}}
\newcommand{\xhat}{\vec{\hat{x}}}
\newcommand{\that}{\hat{t}}

\newcommand{\gammam}{\langle \gamma \rangle}

\newcommand{\Nppc}{N_{\mathrm{ppc}}}

\newcommand{\f}[1]{\ensuremath{f_{\mathrm{ {#1} }}}}
\newcommand{\dt}{\ensuremath{\Delta t}} %delta t
\newcommand{\dd}{\mathrm{d}}


%spesific additions
\newcommand{\runko}{\textsc{Runko}}

\newcommand{\corgi}{\textsc{corgi}}
%\newcommand{\python}{\textsc{Python3}}
\newcommand{\tristan}{\textsc{Tristan}}
\newcommand{\tristanmp}{\textsc{Tristan-MP}}
\newcommand{\pybind}{\textsc{Pybind11}}
\newcommand{\eigen}{\textsc{Eigen}}
\newcommand{\numpy}{\textsc{Numpy}}
\newcommand{\scipy}{\textsc{Scipy}}
\newcommand{\matplotlib}{\textsc{Matplotlib}}


\newcommand{\ene}[2]{\mathcal{E}_{#1}^{#2}} %general energy function
\newcommand{\tcascd}{t_\mathrm{cascd}} %cascade time scale

\newcommand{\todo}[1]{\red{#1}}

\received{MMM D, YYYY}
\revised{MMM D, YYYY}
\accepted{\today}


\shorttitle{MHD turbulence}
\shortauthors{N\"attil\"a}

%________________________________________________________________
\begin{document}


\title{MHD turbulence}

\author{J.~\,N\"attil\"a}
\affiliation{Physics Department and Columbia Astrophysics Laboratory, Columbia University, 538 West 120th Street, New York, NY 10027, USA}
\affiliation{Center for Computational Astrophysics, Flatiron Institute, 162 Fifth Avenue, New York, NY 10010, USA}
\email{email: jan2174@columbia.edu}



\begin{abstract}
    Short review on MHD turbulence theories.
    We start by defining the needed mathematical tools behind many of the turbulence theorices.
    Next, we proceed by presenting a simple Kolmogorov fluid turbulence picture ala K41 model.
    This is followed by a summary of incompressible MHD equations.
    Perturbing these equations leads naturally to the weak turbulence theory.
    As the assumptions behind the weak turbulence break down, we transition to the unknown strong turbulence regime.
    Several phenomenological models are reviewed and physical picture behind some established models are reviewed:
    As a basis we present the Kolmogorov model, its naive extension to MHD by Iroshnikov-Kraichan, and its anisotropic generalization by Goldreich \& Sridhar.
    Related concepts discussed include different cascade time scales, concept of critical balance, and possible dynamical alignement as a cure of intermittency.
    Text is heavily based on review by Schekochihin 2018 (MHD Turbulence: A Biased Review), Verma 2004 (Statistical theory of magnetohydrodynamic turbulence), Beresnyak 2019 (MHD Turbulence), and Biskamp 2003 (Magnetohydrodynamic Turbulence).
\end{abstract}


\keywords{ Plasmas -- Turbulence}

%\maketitle

%________________________________________________________________
%\section{Introduction}\label{sec:intro}
%\citep{birdsall1985}


\section{Statistical background of turbulence cascades}\label{sect:turb}

Single realization of a turbulent flow is chaotic and unpredictable.
Some order can be gained by describing the system statistically by taking ensemble averages of different quantities (averaging over many realizations of the same systems).
According to ergodic hypothesis statistical ensemble averages are replacable by time- or volume-averaging;
this is under the assumption that turbulence is volume-filling and persistent process.

A spatial average of a quantity $u$ is defined as
\be
\tilde{u} = \frac{1}{V_\Omega} \int_\Omega u(\vec{x}) d\vec{x},
\ee
where $V_\Omega$ denotes the volume of $\Omega$.
Temporal averaging is similarly $\bar{u} = \lim_{T\rightarrow\infty} \frac{1}{T} \int_0^T u(t) dt$.
Thanks to the ergodicity we typically use $\tilde{\ldots}$ (or $\bar{\ldots}$) to mimick the role of full ensemble average $\langle \ldots \rangle$.

Fourier transforms of a turbulent quantity $u$ are defined as
\begin{align}\begin{split}
%u(\vec{\ell}) = \int e^{i \vec{k} x} d\hat{u}(\vec{k}). %one component
            \vec{u}(\vec{x}) &= \int \frac{ d\vec{k} }{(2\pi)^D} ~ \hat{\vec{u}}(\vec{k}) e^{i \vec{k} \cdot \vec{x} }, \\
    \hat{ \vec{u}} (\vec{k}) &= \int d\vec{x}~ \vec{u}(\vec{x}) e^{-i \vec{k} \cdot \vec{x} },
\end{split}\end{align}
where $D$ is the space dimensionality.
Note that the transformation is performed independently for each vector component.

Fourier transformed $\langle \hat{u}(\vec{k}) \rangle$ can be related to the so-called energy spectrum $E(k)$.
It describes energy in the wavenumber space such that $dE = E(k) dk$ is the energy at that particular wavenumber (or scale).
Total energy is obtained by summing over all scales, $\ene{}{} = \int E(k) dk$.
For a statistically self-similar system we expect a power-law scaling for $E(k)$.

Square of the Fourier transform is called a power spectrum (giving power spectral density),
\be
P(\vec{k}) d\vec{k} = \langle | \hat{u}(\vec{k}) |^2 \rangle.
\ee
Energy here should be viewed in a generalized sense of the $L^2$ norm, $||\ldots||^2$.
For isotropic system, i.e., if $P(\vec{k})$ depends only on $|k|$, we can integrate over the shells in k-space to get the energy spectrum
\be
E(k) = 4\pi k^2 P(k)
\ee

More general statistical quantities like structure functions and correlation functions are discussed in Sect.~\ref{sect:adv_stat}.

%With the help of energy scale distribution we can define the energy carrying scale.
%This statistical outer scale of the system can be obtained as
%\be
%L_0 = \frac{3\pi}{4 E} \int_0^\infty k^{-1} E(k) dk
%\ee



\section{Basic theoretical background of turbulent cascades}\label{sect:theory}

Energy dissipation for incompressible flow can be defined per unit mass as $\varepsilon$ with units $\cm^2 \second^{-3}$.
Kolmogorov model assumes that the statistical properties of turbulence are uniquely determined by the amount of energy available in a stationary homogenous system, i.e. by $\varepsilon$ alone.
Furthermore, it assumes that energy self-similarly cascades through series of scales known as the inertial range.
Cascade means that energy is being transferred from one scale to another without dissipation.

Kolmogorov model can be deduced from dimensional analysis alone.
If the spectrum, $E(k)$, has units of $\cm^3 \second^{-2}$ and wavenumber, $k = 2\pi/\ell$, has units of $\cm^{-1}$ then we must have
\be
E(k) = C_K \varepsilon^{2/3} k^{-5/3},
\ee
where $C_K$ is a dimensionless Kolmogorov constant.
The $3D$ powerspectrum $P(k) \sim E(k) k^{-2} \sim k^{-11/3}$.

The physics of this derivation is the following.
Characteristic velocity on scale $\ell$ is $u_\ell$ and if the energy rate is constant for all scales then
\be
\frac{u_\ell^2}{\tcascd} = \varepsilon,
\ee
where $\tcascd$ is a cascading time, time it takes for nonlinearities to remove energy from scale $\ell$ and transfer it to smaller scales.

For hydrodynamic cascade we assume that $\tcascd$ is a dynamical time on a particular scale, so that 
\be
\tcascd \approx \frac{\ell}{u_\ell}.
\ee
By substitituion we then have
\be
\varepsilon \sim \frac{u_\ell^2}{\tcascd} \sim \frac{u_\ell^3}{\ell}
\ee
that results in
\be
u_\ell \sim (\varepsilon \ell)^{1/3} \sim \varepsilon^{1/3} k^{-1/3}.
\ee
From the definition of the energy spectrum (band-integrated over Fourier spectrum) we have that 
%$E(k) k \sim u_\ell^2$ 
\be
u_\ell^2 \sim E_\ell \sim \int_{k_\ell}^{k_{\ell+1}} E_k dk \sim E_k k_n
\ee
so that
\be\label{eq:K41}
E(k) \sim \frac{u_\ell^2}{k} \sim \frac{(\varepsilon^{1/3} k^{-1/3})^2}{k} \sim \varepsilon^{2/3} k^{-5/3}.
\ee
This agrees with the result of the dimensional derivation above.

The energy is expected to cascade down to smaller scales until the so-called Kolmogorov scale.
At this scale dissipative processes overcome nonlinear transfer of energy.
Given a diffusivity $\nu_D$ we obtain a measure of a characteristic lenght as
\be
\ell_\nu = \left( \frac{\nu_D^3}{\varepsilon} \right)^{1/4}.
\ee

A critisism against Kolmogrov models is that it does not take into account intermittency.
An empirical correcction is possible by multiplying RHS of Eq. \eqref{eq:K41} with a so-called intermittency correction factor $(k L_0)^\alpha$ where $\alpha \approx 0.035$ (see Frisch 1995).


\section{Alfvenic MHD turbulence}\label{sect:mhd_theory}

Incompressible MHD equations%
\footnote{
    Viability of incompressibility requires the fluid compressibility to be small in the inertial range.
    This assumes that 
    1) turbulence has no shocks
    2) no sizable energy is carried by sound waves (i.e. fast MHD mode),
    3) sonic Mach number, $\mathcal{M}_s = V_L/c_s$ is small and decreases with scale.
}
consist of two dynamical equations and two constraints:
Momentum equation,
induction equation,
divergence free constraint of magnetic field,
and divergence free constraint of velocity field.
These are expressed as
%\begin{align}\begin{split}
%    \partial_t \vec{v} &= -\nabla P/\rho - \nabla (v^2/2) - (\nabla \times \vec{v}) \times \vec{v} + (\nabla \times \vec{b}) \times \vec{b}, \\
%    \partial_t \vec{b} &= \nabla \times (\vec{v} \times \vec{b}).\\
%    \nabla \cdot \vec{b} &= 0\\
%    \nabla \cdot \vec{v} &= 0.
%\end{split}\end{align}
\begin{align}\begin{split}
    \frac{\partial \vec{u}}{\partial t} + \vec{u} \cdot \nabla \vec{u} &= -\nabla P + \vec{B} \cdot \nabla \vec{B} + \nu \nabla^2 \vec{u} \\
    \frac{\partial \vec{B}}{\partial t} + \vec{u} \cdot \nabla \vec{B} &= \vec{B} \cdot \nabla \vec{u} + \eta \nabla^2 \vec{B} \\
    \nabla \cdot \vec{B} &= 0\\
    \nabla \cdot \vec{u} &= 0.
\end{split}\end{align}
Here $\vec{u}$ is velocity, $\vec{B} = \vec{B}_0 + \vec{b}$ is magnetic field (mean $+$ fluctuations), $P$ is total pressure (thermal $+$ magnetic), $\nu$ is kinematic viscosity, $\eta$ is magnetic diffusivity.
Magnetic field is expressed in units of Alfven speed, $b = B_0/\sqrt{4\pi \rho}$.

The above equation can be compressed further by presenting it in terms of the so-called Els\"asser variables, $\vec{z}^\pm = \vec{u} \pm \vec{b}$.
They represent amplitudes of Alfvenic fluctuations with positive and negative correlations.
Note that no pure wave exist in turbulent medium but the interaction can be conveniently written in terms of these variables.
Then
\be
    \frac{\partial \vec{z}^\pm}{\partial t} \mp (\vec{B}_0 \cdot \nabla) \vec{z}^\pm + (\vec{z}^\mp \cdot \nabla) \vec{z}^\pm = -\nabla P  + \nu_- \nabla^2 \vec{z}^\pm  + \nu_+ \nabla^2 \vec{z}^\mp,
\ee
where $\nu_\pm \frac{1}{2} (\nu \pm \eta)$.
Importantly, nonlinear interactions are seen to occur as Alfvenic fluctuations of $z^\mp$ via the nonlinear advective derivative $(\vec{z}^\mp \cdot \nabla) \vec{z}^\mp$.

%The first non-linear term $(\vec{B}_0 \cdot \nabla)\vec{z}^\pm \propto (\vec{v}_A \cdot \nabla)\vec{z}^\pm$ is written in k-space as $(k_\parallel v_A) z^\pm$.
%The second non-linear term is, similarly, $(\vec{z}^\mp \cdot \nabla)\vec{z}^\pm$ expressed as $(k_\perp z^\mp) z^\pm$.
%Turbulence is weak when the first term dominates, $k_\parallel v_A \gg k_\perp z^\mp$; 
%it becomes strong when $k_\parallel v_A \sim k_\perp z^\mp$.
%
%Ideal system conserves Els\"asser variables
%\be
%E^\pm = \int (\vec{z}^+)^2 d\vec{x}
%\ee
%that corresponds to conservation of total energy, $E_\mathrm{tot}$ and cross-helicity $H_\mathrm{C}$ as
%\begin{align}\begin{split}
%    E_{\mathrm{tot}} &= \frac{1}{2} \int u^2 + b^2 d\vec{x} \\
%    H_{\mathrm{C}}   &=             \int \vec{u}\cdot \vec{b}  d\vec{x}.
%\end{split}\end{align}


Alfven fluctuations are therefore characteristic modes of MHD equations.
This means that all perturbations want to resemble Alfvenic fluctuations.
A typical picture of MHD turbulence is a mutual interaction of (multiple) Alfven wave packets moving with Alfven velocities along the guide field.
The interaction of these wave packets can either be weak (small perturbations of waves) or strong (non-linear quick destroyment of wave coherence during a dynamical time).
It is the Alfvenic part of the MHD perturbations that govern this anisotropic turbulence. 
Hence, the name Alfvenic turbulence.



\section{Weak wave turbulence}

Weak turbulence refers to the picture where wave packets propagate almost freely and collisions between the waves leads to small perturbations in their structure so that perturbation theory is applicable.
Mathematically, this takes place in the limit of small $\vec{\delta z}$ that presents a perturbation traveling along the $\vec{B}_0$.
The nonlinear term describes the interaction of the perturbations.
Notably, self-interaction of $\vec{\delta z}^-$ or $\vec{\delta z}^+$ is absent.

The dominant nonlinear interaction is a three-wave process.
Dispersion relation and conservation laws for energy and momentum of the waves is therefore,
\begin{align}\begin{split}
    \omega_n &= k_{\parallel,n} v_A \\
    \pm \omega_1 &= \pm \omega_2 \pm \omega_3 \\
    k_{\parallel,1} &= k_{\parallel,2} + k_{\parallel,3} \\
    k_{\perp,1} &= k_{\perp,2} + k_{\perp,3} 
\end{split}\end{align}
One $\omega_n$ must be zero so let us select $\omega_3=0$ that corresponds to $|k{\parallel,1}| = |k_{\parallel,2}|$.
There is no restriction on $k_{\perp,1,2}$.
The cascade takes place by increasing $k_\perp$ only and preserves the frequencies.

In wave turbulence theory the interaction strenght is
\be
\chi = \frac{k_\perp \delta z}{k_\parallel v_A},
\ee
that describes the ratio of nonlinear shear rate $k_\perp \delta z$ to the wave frequency, $k_\parallel v_A$.
It also servs as an estimate of the nonlinear term to the mean-field term.

In MHD turbulence the dynamical time scale, $t_\mathrm{dyn} = 1/k_\perp \delta z$ (inverse of shear rate), does not have to be proportional to the cascade time $\tcascd$, as is the case for hydrodynamical turbulence.
Instead, the cascade time is increased  by factors of $1/\chi$.

Another way to understand this is to think wave packet perturbations as a random walk.
Each individual perturbation is $\chi$ in strenght so it takes $(1/\chi)^2$ steps to destroy the coherence of a wavepacket.
Then the cascade time is
\be
\tcascd = \frac{1}{k_\parallel v_A} \left( \frac{1}{\chi} \right)^2 = \frac{k_\parallel v_A}{(k_\perp \delta z)^2}
\ee

Energy cascade rate, $\varepsilon$, is the energy on each scale divided by the cascade time on that scale.
This rate is expected to be constant through the inertial scale so that
\be
\epsilon = \delta z^2 \frac{(\delta z k_\perp)^2}{v_A k_\parallel}.
\ee
Since the perturbation is weak, $k_\parallel$ is a constant here.
Then the phenomenological cascade rate is determined by $\delta z^2 \sim k_\perp^{-1}$.
This corresponds to a one-dimensional perpendicular spectrum
\be
E(k_\perp) \sim \delta z^2 k_\perp^{-1} \sim k_\perp^{-2}.
\ee
See Zakharov 1992, Galtier et al 2000, 2002.


Such a weak turbulence model is unstable.
Analytic work has demonstrated that MHD turbulence tends to become stronger and not weaker during a cascade (Galtier et al 2000).
Weak wave turbulence is expected to grow anisotropic with $k_\perp/k_\parallel \sim k_\perp$ since $k_\parallel$ is a constant.
The perturbation strenght, $\chi$ is also expected to become stronger at small scales.
This is seen from
\be
\chi = \frac{\delta z k_\perp}{v_A k_\parallel} \sim k_\perp^{1/2} \rightarrow \infty ~ \mathrm{when }~ k_\perp \rightarrow \infty.
\ee



\section{Phenomenological strong MHD turbulence models}\label{sect:theory}

Before we start with phenomenological strong MHD models, let us make a remark of the actual observations.
Especially the hydrodynamical turbulence is a well-studied problem.
Our best understanding of this system is still roughly a Kolmogorov-like system that predicts $-5/3$ scaling.

In reality, physical and numerical experiments have never agreed with this theoretical scaling.
It is a solid fact that the scaling is more close to $-1.7$ \red{CHECK} and not $-1.66$ as would be predicted.
The situation is even worse for MHD cascades since we can not perform detailed physical experiments of plasma turbulence nor can we simulate large enough MHD systems.%
\footnote{
    The problem is computationally very demanding.
From the simulation perspective, there is a roughly an one order of magnitude of settling scales below the injection scale $\ell_0$ so that inertial range begins at scales of $\lesssim l_0/10$.
Similarly, dissipation (and grid scale) physics (and numerics) are present at roughly one order of magnitude above the dissipation scale $\ell_\nu$ so that a clean inertial range extends to $\gtrsim 10 \ell_\nu$.
In order to perform scaling measurements with a inertial range of length at least 2 decades, we then need to simulate a turbulent system of a size of at least $(10\times10\times10^2)^3 \sim (10^4)^3$.
Current simulations are only starting to probe these scales whereas such a system is only the minimum viable numerical experiment.
}
Therefore, one should not get too carried away by the different theoretical models;
none of them is actually correct.


\subsection{Kolmogorovian theory}\label{sect:K41}

Energy is pumped into a homogenous conducting medium with a fixed rate $\epsilon$.
Dimensionless analysis gives for the energy spectrum \citep{Kolmogorov_1941}
\be
E(k) \sim \epsilon^{2/3} k^{-5/3}.
\ee
Same can be written in terms of average velocity increments
\be
\delta u_\lambda \sim (\epsilon \lambda)^{1/3}.
\ee

\subsection{Iroshnikov-Kraichan theory}\label{sect:K65}

If the $\vec{B}$ field has an important role in energy transfer, then similar dimensionless analysis gives (Iroshnikov Kraichnan 1965)
\be
E(k) \sim (\epsilon v_A)^{1/2} k^{-3/2}
\ee
and
\be
\delta u_\lambda \sim (\epsilon v_A \lambda)^{1/4},
\ee
for the Alf\'ven speed with density $\rho$
\be
v_A = \frac{B}{\sqrt{4\pi \rho}}.
\ee
The reasoning is based on the fact that Alf\'ven time, $\tau_A \sim 1/k v_A$ is the time which interactions occur so the energy must come with a combination $\epsilon v_A$.

The theory is incorrect because it assumes a uniform scale $k$ whereas in reality in a presence of a strong guide field $\vec{B}_0$ the scales split into $k_{\parallel}$ and $k_{\perp}$.
Therefore, the time scale in the assumption is wrong.


\subsection{Goldreich-Sridhar and critical balance}\label{sect:GS95}

In a strong magnetic field $k_{\parallel} \ll k_{\perp}$.
Parallel direction propagation velocity corresponds to Alf\'ven waves with 
\be
\tau_A = \frac{l_{\parallel}}{v_A},
\ee
whereas perpendicular variation is governed by nonlinear intearctions with characteristic times
\be
\tau_{\mathrm{nl}} \sim \frac{l_{\perp}}{\delta u_{\lambda}}.
\ee

For Alf\'venic perturbations $\delta u_{\lambda} \sim \delta b_{\lambda}$.
The two times, $\tau_A$ and $\tau_{\mathrm{nl}}$, are assumed to be equal.
The natural ``cascade time'' must also be of same order, $\tau_c \sim \tau_A \sim \tau_{\mathrm{nl}}$
This gives 
\be
\frac{\delta u_{\lambda}^2}{\tau_c} \sim \epsilon \quad\mathrm{and}\quad
\tau_c \sim \tau_{\mathrm{nl}} \sim \frac{\lambda}{\delta u_{\lambda}},
\ee
so that
\be
\delta u_{\lambda} \sim (\epsilon \lambda)^{1/3}
\ee
and equally \citep{Goldreich_1995, 1997}
\be
E(k_\perp) \sim \epsilon^{2/3} k_{\perp}^{-5/3},
\ee
yielding a Kolmogorov-like scaling for the perpendicular scales.

Simultaneously, along the field the velocity increment satisfy
\be
\frac{\delta u_{\parallel}^2}{\tau_c} \sim \epsilon \quad\mathrm{and}\quad
\tau_c \sim \tau_A \sim \frac{l_{\parallel}}{v_A}
\ee
so that
\be
\delta u_{l \parallel} \sim \left( \frac{\epsilon l_{\parallel}}{v_A} \right)^{1/2}.
\ee
From here it follows
\be
l_{\parallel} \sim v_A \epsilon^{-1/3} \lambda^{2/3}.
\ee
Physically $l_{\parallel}$ is the distance an Alfv\'enic pulse travels along the field at speed $v_A$ over time $\tau_{\mathrm{nl}}$.


\subsection{Remark on weak turbulence}


Weak turbulence theory stems from a perturbation in a (assumedly) small ratio $\tau_A/\tau_{\mathrm{nl}}$.
WT scaling originates from
\be
\delta z_{\lambda} \sim \left( \frac{\epsilon}{\tau_A} \right)^{1/4} \lambda^{1/2}
\ee
where $\delta z_{\lambda}$ is perturbed Elsasser field.
This gives a scaling
\be
E(k_{\perp}) \sim \left( \frac{\epsilon}{\tau_A} \right)^{1/2} k_{\perp}^{-2}.
\ee

Eventually weak turbulence will transition to strong turbulence.
For balanced turbulence this happens when the perturbation parameter becomes of order unity,
\be
\frac{\tau_A}{\tau_{\mathrm{nl}}} \sim \frac{\tau_A^{3/4} \epsilon^{1/4}}{\lambda^{1/2}} \sim 1
\ee
corresponding to a scale (assuming critical balance)
\be
\lambda_{\mathrm{CB}} \sim \epsilon^{1/2} \tau_A^{3/2}.
\ee


\subsection{Critical balance}

In a strong turbulence, $2D$ structures with $\tau_{\mathrm{nl}} \ll \tau_A$ are unsustainable because of causality:
Information propagates along $\vec{B}$ at $v_A$ so no structure longer than $l_{\parallel} \sim v_A \tau_{\mathrm{nl}}$ can be kept coherent (Boldyrev 2005) \citep{Boldyrev_2005}.


Alfv\'en wave is a basic element of MHD motion.
Because of this, strong magnetic perturbations would ``want'' to resemble Alfv\'ven waves as closely as possible.
Critical balance relates to this:
An Alfv\'enic perturbation decorrelates in roughly one wave period.

\subsection{Dynamical alignement}

Dynamic alignment derives from the same assumption of MHD tendency towards Alfv\'enic nature.
In an Alfv\'en wave $\vec{u}_\perp$ and $\vec{b}_\perp$ are the same, i.e., plasma flows drag field with them or the field accelerates flows to relax under tension.
However, if the two fields are exactly parallel, there would be no non-linearity.

Boldyrev's theory on dynamical alignement states that the angle, $\theta_{\lambda}$ between the two fields can not be known more precisely than 
\be
\sin \theta_{\lambda} \sim \theta_\lambda \sim \frac{\delta b}{v_A} \ll 1
\ee
Then the non-linear time is modified to be
\be
\tau_{\mathrm{nl}} \sim \frac{\lambda}{\delta z_\lambda^\mp \sin\theta_\lambda}
\ee
This leads to Kraichnan-type of $-3/2$ scaling
\be
E(k_\perp) \sim (\epsilon v_A)^{1/2} k_\perp^{-3/2}
\ee
and 
\be
l_\parallel \sim v_A^{3/2} \epsilon^{-1/2} \lambda^{1/2}.
\ee

Yet another way to rewrite the scaling relations in terms of the critical balance scale, $\lambda_{\mathrm{CB}}$, is 
\be
E(k_\perp) \sim \epsilon^{2/3} \lambda_{\mathrm{CB}}^{1/6} k_\perp^{-3/2},
\ee
that follows the prediction for the spectrum by Perez et al 2012, 2014b.
This is sometimes known as aligned turbulence.


\subsection{Intermittency}

Classical turbulence theories rely on self-similarity of the structures. 
Intermittency means that this assumption is broken and instead we introduce all three length scale directions: 
perpendicular $\lambda$, parallel $l_\parallel$, and fluctuation direction $\zeta$.

Intermittency states that eddies are not completely space-filling, but have more rare fluctuations on top of the ``typical'' ones.
In hydrodynamic turbulence, corrections to K41 theory come in powers of $\lambda/L$.
Similar can be found in MHD turbulence as the self-similarity is broken with the appeareance of outer-scale size $L_\parallel$ in to the scaling equations.

Mallet \& Schekochihin conjectured that $l_\parallel \sim \lambda^\alpha$, i.e., $l_\parallel/\lambda^\alpha$ has a scale-invariant distribution.
A typical (second) conjecture is that most intense structures are sheets transverse to the local perpendicular direction.

%Note: Some arguments about outer scale structures lead to a prediction on MS17 for alpha=1/2.

\subsection{Phenomenological unified picture of Schekochihin}

In reality, the scaling are expected to be fully $3D$.
From critical balance and from the assumption of dynamical alignment it then follows that that field's spectra depends as:
$-2$ in $l_\parallel$ direction,
$-3/2$ in $\lambda$ direction, and
$-5/3$ in $\zeta$ direction.

In addition, we have
\be
\zeta \sim l_\parallel \frac{\delta z_\lambda}{v_A} \sim l_\parallel \frac{\delta b_\lambda}{v_A}a,
\ee
i.e., $\zeta$ is the typical displacement of a fluid element and a typical perpendicular distance a field line wanders within a structure that is coherent on the parallel scale $l_\parallel$.
Fluctuations must therefore preseve coherence in their respective direction at least on scale $\zeta$.
They are not constrained in the third $\lambda$ direction.
Finally, they are expected to have an angular uncertainty of the order of the angle $\theta_\lambda$ between the two fields.

Finally, we note that this is picture is not solid yet.
On the contrast, it relies on the refined dynamical alignment conjecture that itself is not proven to be correct or incorrect yet.
Another way to think about this conjecture is, that it is used to describe the intermittency.
For a solid theory of MHD turbulence, a complete theory of intermittenty is therefore needed.



\section{Deviations from incompressibility and non-relativistic assumptions}\label{sect:kin_theory}

\red{TODO: deviations from above: compressibility, relativistic motions, }

Astrophysical plasmas that are magnetically dominated and undergo a strong non-linear driving can show various deviations from the above assumptions.
Most notably, these include relativistic bulk motions that can induce a dynamically important electric field.
For highly conductive flows $E \approx \frac{u}{c} B$.
For non-relativistic flows electric field remains small, $E \ll B$.
If the driving scale motions are relativistic, the electric field can become large enough to become dynamically important itself.

Same is true for the current as the plasma can become charge starved.
From Maxwell's equations we have that
\be
\nabla \times \vec{B} = \frac{4\pi}{c} \vec{J} + \frac{1}{c} \frac{\partial E}{\partial t}.
\ee
For non-relativistic plasma we can ignore the last term since it is $\propto (u/c)^2$.
Therefore, 
\be
\vec{J} = \frac{c}{4\pi} \nabla\times\vec{B}.
\ee
Hence, both $\vec{E}$ and $\vec{J}$ are dependent variables.
For relativistic flows current can become charge starved and become dynamically imporant too.

Relativistic bulk motions can also lead to introduction of compressible modes.
For magnetized plasmas Alfven speed, $v_A = B/\sqrt{4\pi\rho}$ plays the role of sound speed in the medium.
The fluid is incompressible if $u \ll v_A$.
For relativistic plasma with strong non-linear driving strenght both velocities become comparable, $u \sim v_A$.


Magnetized plasma can support a set of different wave modes.
Most notable is the Alfven wave that is incompressible mode.
For Alfven wave $u_\parallel = u_\perp = b_\perp = 0$ but $\delta u_\perp \ne 0$ and $\delta b_\perp \ne 0$.


Cho and Lazarian (2002, 2003) suggest that Alfven mode has an independent cascade, slow mod eis passive to Alfven cascade, and has the same spectra and isotropy, and fast mode has an independent isotropic acoustic/wave turbulence cascade.
Kowal and Lazarian 2010 report steeper $k^{-2}$ spectra for fast modes.

Realistic compressible turbulence is strongly dependent on the way it is driven;
this defines the initial wave content and degree of compressibility.

Density is primarily perturbed by the slow mode.
In a magnetized plasma with slow mode perturbations the slow mode is mainly along $\vec{B}_0$.
Beresnyak 2005
Alfven modes only mix these perturbations by shearing motions.

See decaying MHD turbulence that is driven by initial random magnetic fields (e.g. Biskamp 2003) that is qualitatively similar to reconnection turbulence.


Coupling between fluid and magnetic field gives field lines stiffness and elasticity.
This turns unstable perturbations into propagating Alfven waves.
The coupling is not perfect since resistivity allows some slippage of field across the fluid can take place.
This allows one to tap to the reservoir of free magnetic energy more efficiently.

The driving mechanism is the attractive force of parallel current elements.


\section{More advanced statistical quantities}\label{sect:adv_stat}

\subsection{Structure functions}

Structure functions are averages of differences between turbulent quantities measured at different nearby locations.
General order $p$ structure function of a physical variable $\vec{X}(\vec{x})$ (like $\vec{u}$, $\vec{B}$, etc.) are defined as
\be
S^X_p(r) = \langle [ X(\vec{x}-\vec{r}) - X(\vec{x}) ]^p \rangle,
\ee
where $\langle \ldots \rangle$ denotes averaging (time, volume, ensemble). %; no stance taken on what the exact nature actually is).
This quantity captures spatial correlation of the field $\vec{X}$.
%For velocity it reduces to kinetic energy, $S^u = 4 E_K$ on a given scale.

Physically structure functions can be thought of as a measures of gradients, since Taylor expanding
\be
u(\vec{x} + \vec{r}) - u(\vec{x}) \sim \frac{\partial u}{\partial x} r.
\ee
Especially useful is the second order structure function, $S_2$ that can be thought of as a measure of energy fluctuations,
\be
S_2^u(r) = \langle [ u(\vec{x}-\vec{r}) - u(\vec{x}) ]^2 \rangle.
\ee
Indeed, spectra and structure functions actually have one-to-one correspondence
\be
S_2(\ell) = 2\int_0^\infty \left( 1 - \frac{\sin kr}{kr}\right) E(k) dk
\ee
If spectrum is a power-law, $E(k) \propto k^\alpha$, then by substitution of $k = x/\ell$, we have $S_2(\ell) \propto \ell^{-1-\alpha}$.

From a statistical viewpoint, if the turbulence is self-similar (i.e., has a single-fractal structure) higher-order structure functions are all connected as 
\be
[S_n(r))]^{1/n} \sim [S_m(r))]^{1/m},
\ee
for any arbitrary orders $n$ and $m$.

\subsection{Correlation functions}

Correlation functions are typically used to study temporal behavior.
They can, however, be also generalized to spatial correlations.
Similar to structure functions we can connect spectra and correlation functions together.

\red{TODO: Autocorrelation}

%try2 on structure functions
Two-point correlation function is
\be
\langle u_i(\vec{r}) u_j(\vec{r'})\rangle = C_{ij}(\vec{r}-\vec{r}') = C_{ij}(\vec{r})
\ee
Isotropic two-point correlation function reduces to
\be
C_{ij}(\vec{r}) = C^{(1)}(r) r_i r_j + C^{(2)}(r) \delta_{ij}.
\ee
Energy and other second-order quantities play an important role in MHD turbulence.
For homogeneous system these are defined as
\be
\langle X_i(\vec{k}) Y_j(\vec{k}'\rangle = C_{ij}^{XY}(\vec{k}) [2\pi]^D \delta(\vec{k} + \vec{k}')
\ee

The spectrum and correlation function in real space are connected as
\be
E^X = \frac{1}{2} \langle X^2 \rangle = \frac{1}{2} \int \frac{d\vec{k}}{(2\pi)^D}  C_{ij}^{XY}(\vec{k})
\ee
and
\be
\int E^X(k) dk = \frac{1}{2} \int dk \frac{S_D (D-1)}{(2\pi)^D} k^{D-1} C^{XX}(\vec{k}),
\ee
where $S_D = 2\pi^{D/2} / \Gamma(D/2)$ is the area of $D$-dimensional unit sphere.
Therefore, for isotropic turbulence 
\be
E^X (k) = C^{XX}(\vec{k}) k^{D-1} \frac{S_D (D-1)}{2(2\pi)^D}.
\ee


\section{Literature \red{Work-in-progress}}\label{sect:literature}

Objects themselves:
\begin{itemize}
    \item PWN: Woosley 1993
    \item jets from AGNs: Reynolds 1996
    \item GRBs: Wardle 1998
\end{itemize}

%--------------------------------------------------
%plasma turbulence as a generator of non-thermal particles
\subsection{Non-thermal particles from turbulence}
\begin{itemize}
    \item Melrose 1980;
    \item Petrosian 2012;
    \item Lazarian 2012;
\end{itemize}

\subsection{Turbulence in astrophysics}

turbulence in stellar coronae:
Matthaeus 1999, Cranmer 2007

ISM:
Armstrong 1995, Lithwick \& Goldreich 2001

SNRs:
Weiler \& Sramek 1988, Roy 2009

PWN:
Porth 2014, Lyutikov 2019

BH disks:
Balbus \& Hawley 1998, Brandenburg \& Subramanian 2005

jets from AGNs:
Marscher 2008, MacDonald \& Marscher 2018
\citep{MacDonald_2018}

radio lobes:
Vogt \& Ensslin 2005, O'Sullivan 2009

GRBs:
Wardle 1998
Piran 2004, 
\citep{Kumar_2009}

Galaxy clusters:
Zweibel \& Heiles 1997, Subramanian 2006

Laser laboratory plasma:
Sarri 2015


\subsection{Magnetically-dominated turbulence}

Sustained relativistic turbulence (force-free)
\citep{Thompson_1998}: extension of Goldreich \& Sridhar 1995 to exterme relativistic limit (no plasma inertia; force-free MHD).
Anisotropic cascade is formed, dissipation occurs at the scale of current starvation (when not enough charge carriers in plasma to maintain currents from Alf\'en waves).

MHD:
\citep{Cho_2005}
Inoue 2011
\citep{Cho_2014}
\citep{Zrake_2016}

Relativistic MHD:
\citep{Zrake_2012}
\citep{Zrake_2014}

\subsection{Bright non-thermal synchrotron and inverse Compton signatures}
Pulsar magnetospheres and winds:
Buhler \& Blandford 2014

Jets from AGNs: Begelman 1984

Coronae of accretion disks:
\citep{Yuan_2014}


\subsection{Kinetic turbulence}

\citep{Zhdankin_2017a} Letter\\
\citep{Zhdankin_2017b} Paper\\
\citep{Zhdankin_2018} System size convergence\\
\citep{Zhdankin_2019a} electron-proton plasma\\
\citep{Zhdankin_2019b} radiative turbulence\\
\citep{Comisso_2018} acceleration\\
\citep{Wong_2019} acceleration\\
\citep{Nattila_2019} Runko and turbulence\\
\citep{Comisso_2019} acceleration\\

\subsection{Radiative turbulence}

\subsubsection{Analytic work on radiative turbulence}
\citep{Uzdensky_2018}; \\
\citep{Zrake_2018} (GRBs) \\
\citep{Sobacchi_2019}; \\
 
\subsubsection{PIC simulations}
\citep{Zhdankin_2019b} radiative turbulence\\


\subsubsection{Fokker-Planck equation in momentum space with radiative cooling term}
\citep{Schlickeiser_1984, Schlickeiser_1985}; not in original list \citep{Schlickeiser_1989} \\
\citep{Stawarz_2008} \\


\subsection{radiative PIC simulations}

\subsubsection{Reconnection}
\citep{Jaroschek_2009} \\
\citep{Cerutti_2013, Cerutti_2014, Cerutti_2014a} \\
\citep{Kagan_2016, Kagan_2016b} \\
\citep{Hakobyan_2019} \\
\citep{Werner_2019} \\
\citep{Schoeffler_2019} \\

\subsubsection{Decay of magnetostatic equilibria}
\citep{Yuan_2016} \\
\citep{Nalewajko_2018} \\

\subsubsection{Pulsar wind}
\citep{Cerutti_2017} \\

\subsubsection{Pulsar magnetospheres}
\citep{Cerutti_2016a} \\
\citep{Philippov_2018} \\


\subsubsection{Synchrotron and jitter radiative signatures of collisionless shocks}
\citep{Medvedev_2009} \\
\citep{Sironi_2009} \\
\citep{Kirk_2010} \\
\citep{Nishikawa_2011} \\

\subsubsection{Radiative turbulence}
\citep{Zhdankin_2019}




%\section{Summary}\label{sect:summary}

%\section*{Acknowledgments}
%JN would like to thank XXX stimulating discussions on numerics and plasma simulations. 
%The simulations were performed on resources provided by the Swedish National Infrastructure for Computing (SNIC) at PDC and HPC2N.


\bibliographystyle{aa}
\bibliography{refs}

%\begin{thebibliography}{62}
%\end{thebibliography}

%\clearpage
%\onecolumn
%\begin{appendix}
%\end{appendix}

\end{document}
