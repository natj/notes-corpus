%\documentclass[referee]{aa}
\documentclass[usenatbib,twocolumn, twocolappendix]{aastex63}

\pdfoutput=1
\pdfminorversion=5


\usepackage[utf8]{inputenc}
\usepackage{tabularx}
\usepackage{multirow}
\usepackage{url}
\usepackage{amsmath}	% Advanced maths commands
\usepackage{amssymb}	% Extra maths symbols
\usepackage{graphicx}
\usepackage{color}
\usepackage{natbib}
\usepackage{hyperref}

%\usepackage{newtxtext,newtxmath}
\usepackage{times,txfonts}
\usepackage{bm}

% Depending on your LaTeX fonts installation, you might get better results with one of these:
%\usepackage{mathptmx}
%\usepackage{txfonts}

% Use vector fonts, so it zooms properly in on-screen viewing software
% Don't change these lines unless you know what you are doing
\usepackage[T1]{fontenc}
\usepackage{ae,aecompl}


%landscape figures
%\usepackage{wrapfig}
%\usepackage{lscape}
%\usepackage{rotating}

\bibpunct{(}{)}{;}{a}{}{,} % to follow the A&A style

%Debug addition for collaborators
%\usepackage[switch, modulo]{lineno}
%\linenumbers
%\renewcommand\linenumberfont{\color{red}\normalfont\tiny\sffamily}
%\renewcommand\linenumberfont{\normalfont\tiny\sffamily}

\DeclareUnicodeCharacter{00A0}{ }

\usepackage{listingsutf8}

% configure listings package
\lstset{
         basicstyle=\footnotesize\ttfamily, 
         %numbers=left,               
         numberstyle=\tiny,          
         %stepnumber=2,             
         numbersep=5pt,             
         tabsize=2,                  
         extendedchars=true,         
         breaklines=true,            
         keywordstyle=\color{red},
         frame=b,
 %        keywordstyle=[1]\textbf,   
 %        keywordstyle=[2]\textbf,   
 %        keywordstyle=[3]\textbf,   
 %        keywordstyle=[4]\textbf,   
         %stringstyle=\color{white}\ttfamily, % Farbe der String
         showspaces=false,           
         showtabs=false,            
         xleftmargin=17pt,
         framexleftmargin=17pt,
         framexrightmargin=5pt,
         framexbottommargin=4pt,
         %backgroundcolor=\color{lightgray},
         showstringspaces=false    
}
\lstloadlanguages{C++}
\newcommand{\codeil}[1]{\lstinline!#1!}

\usepackage{amsmath}
\usepackage{amssymb}


%--------------------------------------------------
%Basic commands
\newcommand{\be}{\begin{equation}}
\newcommand{\ee}{\end{equation}}
\newcommand{\bea}{\begin{align}\begin{split}}
\newcommand{\eea}{\end{split}\end{align}}

\newcommand{\fig}[1]{Fig.~\ref{#1}}
\newcommand{\eq}[1]{~(\ref{#1})}
\newcommand{\sect}[1]{Sect.~\ref{#1}}
\newcommand{\appnd}[1]{Appendix \ref{#1}}

\newcommand{\Ten}[2]{\ensuremath{#1 \times 10^{#2}} }


%-------------------------------------------------- 
% Units with proper unit spacing
\newcommand{\unitspace}{\,}

\newcommand{\km}{\ensuremath{\unitspace \mathrm{km}}}
\newcommand{\cm}{\ensuremath{\unitspace \mathrm{cm}}}
\newcommand{\g}{\ensuremath{\unitspace \mathrm{g}}}
\newcommand{\Hz}{\ensuremath{\unitspace \mathrm{Hz}}}
\newcommand{\gcm}{\ensuremath{\unitspace \mathrm{g}\,\mathrm{cm}^{-3}}}
\newcommand{\cms}{\ensuremath{\unitspace \mathrm{cm}\,\mathrm{s}^{-1}}}
\newcommand{\cmss}{\ensuremath{\unitspace \mathrm{cm}\,\mathrm{s}^{-2}}}
\newcommand{\dyncm}{\ensuremath{\unitspace \mathrm{dyn}\,\mathrm{cm}^{-2}}}
\newcommand{\erg}{\ensuremath{\unitspace \mathrm{erg}}}
\newcommand{\ergs}{\ensuremath{\unitspace \mathrm{erg}\,\mathrm{s}^{-1}}}

\newcommand{\Gauss}{\ensuremath{\unitspace \mathrm{G}}}
\newcommand{\Kelvin}{\ensuremath{\unitspace \mathrm{K}}}

%\newcommand{\Msun}{\ensuremath{\unitspace M_{\odot}}}
\newcommand{\Msun}{\ensuremath{\unitspace \mathrm{M}_{\odot}}}


%--------------------------------------------------
% colors
\newcommand{\red}[1]{\textcolor{red}{#1}}
\newcommand{\green}[1]{\textcolor{green}{#1}}
\newcommand{\blue}[1]{\textcolor{blue}{#1}}



%--------------------------------------------------
% vectors

%normal 3-vectors
%\renewcommand{\vec}[1]{\ensuremath{\mathbf{#1}}}
%\renewcommand{\vec}[1]{\ensuremath{\boldsymbol{#1}}}
\renewcommand{\vec}[1]{\ensuremath{\bm{\mathit{#1}}}}

%four-vectors
\makeatletter
\def\fvec#1{\underline{\sbox\tw@{$#1$}\dp\tw@\z@\box\tw@}}
\makeatother


%--------------------------------------------------
% math

% integrals
\newcommand{\ud}{\mathrm{d}} % differential d

% derivatives
\newcommand{\pD}[2]{\ensuremath{\partial_{#1}}} %partial derivative
\newcommand{\fracpD}[2]{\ensuremath{\frac{\partial {#1} }{\partial {#2} }}} %frac partial

\newcommand{\derfrac}[2]{\frac{\ud #1}{\ud #2}} % fractional deriv
\newcommand{\parfrac}[2]{\frac{\partial #1}{\partial #2}} %fractional partial deriv
\newcommand{\parD}[1]{\partial_{#1}} %partial derivative shorthand

%misc
\newcommand{\osim}{\mathord{\sim}} % ordo O

% statistics
\newcommand{\bra}[1]{\ensuremath{\langle #1 \rangle}}
\newcommand{\braket}[2]{\ensuremath{ \langle #1 ~|~ #2 \rangle}}


%--------------------------------------------------
% programming

% writing c++ is highly non-trivial
%\newcommand{\cpp}{C\nolinebreak\hspace{-.05em}\raisebox{.4ex}{\tiny\bf +}\nolinebreak\hspace{-.10em}\raisebox{.4ex}{\tiny\bf +{rbrace}{rbrace}}}
%\newcommand{\cpp}{C\nolinebreak\hspace{-.05em}\raisebox{.4ex}{\tiny\bf +}\nolinebreak\hspace{-.10em}\raisebox{.4ex}{\tiny\bf +}}
%\def\cpp{{C\nolinebreak[4]\hspace{-.05em}\raisebox{.4ex}{\tiny\bf ++}}}
%\def\CC{lbrace}{lbrace}C\nolinebreak[4]\hspace{-.05em}\raisebox{.4ex}{\tiny\bf ++{rbrace}{rbrace}{rbrace}
%\usepackage{relsize}
%\newcommand\cpp{C\nolinebreak\hspace{-.05em}\raisebox{.4ex}{\relsize{-3}{\textbf{+}}}\nolinebreak\hspace{-.10em}\raisebox{.4ex}{\relsize{-3}{\textbf{+}}}}
%\newcommand\cpp{C\nolinebreak[4]\hspace{-.05em}\raisebox{.4ex}{\relsize{-3}{\textbf{++}}}}

\newcommand{\cpp}{C\texttt{++}}
\newcommand{\cppv}[1]{C\texttt{++}$#1$}

\newcommand{\python}{\textsc{Python}}





\newcommand{\Cfl}{\hat{c}}
\newcommand{\hf}{\frac{1}{2}}

\newcommand{\Bhat}{\vec{\hat{B}}}
\newcommand{\Ehat}{\vec{\hat{E}}}
\newcommand{\Jhat}{\vec{\hat{J}}}
\newcommand{\uhat}{\vec{\hat{u}}}
\newcommand{\vhat}{\vec{\hat{v}}}
\newcommand{\xhat}{\vec{\hat{x}}}
\newcommand{\that}{\hat{t}}

\newcommand{\gammam}{\langle \gamma \rangle}

\newcommand{\Nppc}{N_{\mathrm{ppc}}}

\newcommand{\f}[1]{\ensuremath{f_{\mathrm{ {#1} }}}}
\newcommand{\dt}{\ensuremath{\Delta t}} %delta t
\newcommand{\dd}{\mathrm{d}}


%spesific additions
\newcommand{\runko}{\textsc{Runko}}

\newcommand{\corgi}{\textsc{corgi}}
%\newcommand{\python}{\textsc{Python3}}
\newcommand{\tristan}{\textsc{Tristan}}
\newcommand{\tristanmp}{\textsc{Tristan-MP}}
\newcommand{\pybind}{\textsc{Pybind11}}
\newcommand{\eigen}{\textsc{Eigen}}
\newcommand{\numpy}{\textsc{Numpy}}
\newcommand{\scipy}{\textsc{Scipy}}
\newcommand{\matplotlib}{\textsc{Matplotlib}}


\newcommand{\ene}[2]{\mathcal{E}_{#1}^{#2}} %general energy function
\newcommand{\tcascd}{t_\mathrm{cascd}} %cascade time scale

\newcommand{\todo}[1]{\red{#1}}

\received{MMM D, YYYY}
\revised{\today}
\accepted{MMM D, YYYY}


\shorttitle{MHD turbulence}
\shortauthors{N\"attil\"a}

%________________________________________________________________
\begin{document}


\title{MHD turbulence}

\author{J.~\,N\"attil\"a}
\affiliation{Physics Department and Columbia Astrophysics Laboratory, Columbia University, 538 West 120th Street, New York, NY 10027, USA}
\affiliation{Center for Computational Astrophysics, Flatiron Institute, 162 Fifth Avenue, New York, NY 10010, USA}
\email{email: nattila.joonas@gmail.com}



\begin{abstract}
    These notes provide a short review of magnetohydrodynamic (MHD) turbulence.
    We start by defining the needed mathematical tools behind many turbulence theories.
    Next, we present a simple fluid turbulence picture \'a~la Kolmogorov 1941.
    This is followed by a summary of incompressible MHD equations.
    Perturbing these equations leads naturally to the weak turbulence theory.
    As the assumptions behind the weak turbulence break down, we transition to the unknown strong turbulence regime.
    Several phenomenological models that seek to model the strong turbulence regime are reviewed, and the physical picture behind some established models is discussed:
    as a basis, we present the Kolmogorov model, its na\"ive extension to MHD by Iroshnikov \& Kraichan, and its anisotropic generalization by Goldreich \& Sridhar.
    Related concepts discussed include different cascade time scales, the concept of critical balance, and possible dynamical alignment.
\end{abstract}


\keywords{ Plasmas -- Turbulence}

%\maketitle

%________________________________________________________________
%\section{Introduction}\label{sec:intro}
%\citep{birdsall1985}

\section{Introduction}

Turbulence is often understood as a ``chaotic'' state of a fluid.
The simplest thought experiment we can make is to ask what constitutes a chaotic and non-chaotic configuration?
Non-chaotic systems should depict some kind of order and simplicity---we call such fluid configurations laminar:
laminar flows have smooth and regular flow patterns in them.
A counter-example of an ordered \& ``simple'' configuration can be devised by thinking of a fluid with perturbations on multiple scales.
Such irregular and complex flow patterns are often called turbulent.
While this is not exactly true---there is some order in turbulent flows---it gives a good mental picture to start thinking about turbulence as a physical phenomenon.


\subsection{History (based on Von Neumann's analysis)}

The earliest scientific note of turbulence is by Reynolds in 1883:
he studied the flow of water in a cylindrical pipe and concluded that the flow is laminar below a certain critical velocity.
When the limit was exceeded, the flow became turbulent with highly irregular and rapidly changing streamlines.

Physically, he found that in the laminar state, the pressure gradient is proportional to flow velocity $U$ and (kinematic) viscosity $\nu$.
In the turbulent regime, the system is more akin to a $U^2$ dependency while $\nu$-dependency is less developed.
Reynolds also found empirically that (within some limits) only the combination of $R_e = \frac{L U}{\nu}$ of the variables mattered in determining whether the critical point had been reached.
This is also the only possible dimensionless combination of the above variables.
It is now generally agreed that (if no special precautions are taken) $R_e$ is around $1000$ to $2000$ when the critical transition threshold is reached and the flow becomes turbulent. 


From the mathematical viewpoint, turbulence is a phenomenon of instability.
This means that the laminar solution of Navier-Stokes equations cease to be (at least the most) stable one for high enough $R$.
Turbulent flow represents one or more solutions of higher stability.
In fact, there is no reason to believe that one solution is favored; 
instead, the turbulent solutions represent an ensemble of statistical properties.

Studies of stability theories will only tell when the laminar solution breaks down.
The turbulent properties themselves are only understood by understanding all the nonlinear solutions of the Navier-Stokes equations.
This makes ``solving'' turbulence a very hard problem!

Solutions of Navier-Stokes equations are quite essentially non-linear.
Turbulence is introduced with large $R_e$, hence, its solutions are related to the asymptotic $R_e \rightarrow \infty$.
This corresponds to $\nu \rightarrow 0$.
For no dissipation ($\nu \rightarrow 0$), a bad singularity occurs as the diffusion term ($\nu \nabla^2 (\nabla^2 \phi)$; order 4, degree 1) disappears, and hence the non-linear term ($\mathcal{J}(\nabla^2 \phi, \phi)$; order 3, degree 2) emerges as the dominant.%
\footnote{
    Order refers to differentiation, degree to algebraic character.
    The term is leading the solution by its order.
    Thus, the character of the equation (which is controlled by the leading term by virtue of its higher order of differentiation) changes simultaneously in all relevant aspects (both order and degree change).
}
This already implies that the nature of the solutions must be changed since the underlying mathematical equations change so drastically. 
The mathematical nature of Navier-Stokes equations was studied by Oseen (1914, 1927) and Leray (1933).




The following text, describing the MHD turbulence in more detail, is heavily based on a review by Schekochihin 2018 (MHD Turbulence: A Biased Review), Verma 2004 (Statistical Theory of Magnetohydrodynamic Turbulence), Beresnyak 2019 (MHD Turbulence), and Biskamp 2003 (Magnetohydrodynamic Turbulence).




\section{Statistical background of turbulence cascades}\label{sect:turb}

The single realization of a turbulent flow is chaotic and unpredictable.
Some order can be gained by describing the system statistically by taking ensemble averages of different quantities (averaging over many realizations of the same systems).
According to the ergodic hypothesis, statistical ensemble averages are replaceable by time- or volume-averaging;
this is under the assumption that turbulence is a volume-filling and temporally-persistent process.

A spatial average of a quantity $u$ is defined as
\be
\tilde{u} = \frac{1}{V_\Omega} \int_\Omega u(\vec{x}) d\vec{x},
\ee
where $V_\Omega$ denotes the volume of $\Omega$.
Temporal averaging is similarly $\bar{u} = \lim_{T\rightarrow\infty} \frac{1}{T} \int_0^T u(t) dt$.
Thanks to the ergodicity, we typically use $\tilde{\ldots}$ (or $\bar{\ldots}$) to mimic the role of full ensemble average $\langle \ldots \rangle$.

Fourier transforms of a turbulent quantity $u$ are defined as
\begin{align}\begin{split}
%u(\vec{\ell}) = \int e^{i \vec{k} x} d\hat{u}(\vec{k}). %one component
            \vec{u}(\vec{x}) &= \int \frac{ d\vec{k} }{(2\pi)^D} ~ \hat{\vec{u}}(\vec{k}) e^{i \vec{k} \cdot \vec{x} }, \\
    \hat{ \vec{u}} (\vec{k}) &= \int d\vec{x}~ \vec{u}(\vec{x}) e^{-i \vec{k} \cdot \vec{x} },
\end{split}\end{align}
where $D$ is the space dimensionality.
Note that the transformation is performed independently for each vector component.

Fourier transformed $\langle \hat{u}(\vec{k}) \rangle$ can be related to the so-called energy spectrum $E(k)$.
It describes the energy in the wavenumber space such that $dE = E(k) dk$ is the energy at that particular wavenumber (or scale).
The total energy is obtained by summing over all scales, $\ene{}{} = \int E(k) dk$.
We expect a power-law scaling for $E(k)$ for a statistically self-similar system.

The square of the Fourier transform is called a power spectrum (giving power spectral density),
\be
P(\vec{k}) d\vec{k} = \langle | \hat{u}(\vec{k}) |^2 \rangle.
\ee
Energy here should be viewed in a generalized sense of the $L^2$ norm, $||\ldots||^2$.
For an isotropic system, i.e., if $P(\vec{k})$ depends only on $|k|$, we can integrate over the shells in k-space to get the energy spectrum
\be
E(k) = 4\pi k^2 P(k)
\ee

More general statistical quantities like structure functions and correlation functions are discussed in Sect.~\ref{sect:adv_stat}.

%With the help of energy scale distribution, we can define the energy-carrying scale.
%This statistical outer scale of the system can be obtained as
%\be
%L_0 = \frac{3\pi}{4 E} \int_0^\infty k^{-1} E(k) dk
%\ee



\section{Basic theoretical background of turbulent cascades}\label{sect:theory}

Energy dissipation for incompressible flow can be defined per unit mass as $\varepsilon$ with units $\cm^2 \second^{-3}$.
Kolmogorov model assumes that the statistical properties of turbulence are uniquely determined by the amount of energy available in a stationary homogenous system, i.e., by $\varepsilon$ alone.
Furthermore, it assumes that energy self-similarly cascades through a series of scales known as the inertial range.
Cascade means energy is transferred from one scale to another without dissipation.

Kolmogorov model can be deduced from dimensional analysis alone.
If the spectrum, $E(k)$, has units of $\cm^3 \second^{-2}$ and wavenumber, $k = 2\pi/\ell$, has units of $\cm^{-1}$ then we must have
\be
E(k) = C_K \varepsilon^{2/3} k^{-5/3},
\ee
where $C_K$ is a dimensionless Kolmogorov constant.
The $3D$ powerspectrum $P(k) \sim E(k) k^{-2} \sim k^{-11/3}$.

The physics of this derivation is the following.
Characteristic velocity on scale $\ell$ is $u_\ell$ and if the energy rate is constant for all scales, then
\be
\frac{u_\ell^2}{\tcascd} = \varepsilon,
\ee
where $\tcascd$ is a cascading time, the time it takes for nonlinearities to remove energy from scale $\ell$ and transfer it to smaller scales.

For hydrodynamic cascade, we assume that $\tcascd$ is a dynamical time on a particular scale so that 
\be
\tcascd \approx \frac{\ell}{u_\ell}.
\ee
By substitution, we then have
\be
\varepsilon \sim \frac{u_\ell^2}{\tcascd} \sim \frac{u_\ell^3}{\ell},
\ee
that results in
\be
u_\ell \sim (\varepsilon \ell)^{1/3} \sim \varepsilon^{1/3} k^{-1/3}.
\ee
From the definition of the energy spectrum (band-integrated over the Fourier spectrum) we have that 
%$E(k) k \sim u_\ell^2$ 
\be
u_\ell^2 \sim E_\ell \sim \int_{k_\ell}^{k_{\ell+1}} E_k dk \sim E_k k_n
\ee
so that
\be\label{eq:K41}
E(k) \sim \frac{u_\ell^2}{k} \sim \frac{(\varepsilon^{1/3} k^{-1/3})^2}{k} \sim \varepsilon^{2/3} k^{-5/3}.
\ee
This agrees with the result of the dimensional derivation above.

The energy is expected to cascade to smaller scales until the so-called Kolmogorov scale.
At this scale, dissipative processes overcome the nonlinear transfer of energy.
Given a diffusivity $\nu_D$, we obtain a measure of a characteristic length as
\be
\ell_\nu = \left( \frac{\nu_D^3}{\varepsilon} \right)^{1/4}.
\ee

A criticism against the Kolmogorov model is that it does not consider intermittency.
An empirical correction is possible by multiplying RHS of Eq. \eqref{eq:K41} with a so-called intermittency correction factor $(k L_0)^\alpha$ where $\alpha \approx 0.035$ (see Frisch 1995).


\section{Alfvenic MHD turbulence}\label{sect:mhd_theory}


The general picture of fluid turbulence consists of turbulent eddies cascading to smaller scales, enabling the forward transfer of energy in Fourier space. 
But what are the ``magnetic'' eddies?
In MHD turbulence, Alfv\'en waves are thought to be responsible for the nonlinear interaction.
These are not fluid vortices (eddies) but magnetic twists (Alfv\'en waves).

However, Alfv\'en waves, by definition, are linear eigenmodes of the system (i.e., waves) and can not cause the energy transfer.
Instead, the energy transfer is enabled by an inherently nonlinear magnetic shear mode---a local field line twist---produced by the nonlinear interaction of two overlapping orthogonally polarized waves.
This shear is supported by a local electric field current region, $\delta B_\perp \sim \nabla \times J_z$.
Therefore, we can think of these magnetic field shear regions, localized $J_z$ fluctuations, another integral part of MHD turbulence.
Therefore, MHD turbulence should have both eddies (magnetic twists) and intermittent structures (current sheets) that contribute to the energy transfer.



Incompressible MHD equations%
\footnote{
    Viability of incompressibility requires the fluid compressibility to be small in the inertial range.
    This assumes that 
    1) turbulence has no shocks
    2) no sizable energy is carried by sound waves (i.e., fast MHD mode),
    3) sonic Mach number, $\mathcal{M}_s = V_L/c_s$ is small and decreases with scale.
}
consist of two dynamical equations and two constraints:
Momentum equation,
induction equation,
divergence-free constraint of magnetic field,
and divergence-free constraint of the velocity field.
These are expressed as
%\begin{align}\begin{split}
%    \partial_t \vec{v} &= -\nabla P/\rho - \nabla (v^2/2) - (\nabla \times \vec{v}) \times \vec{v} + (\nabla \times \vec{b}) \times \vec{b}, \\
%    \partial_t \vec{b} &= \nabla \times (\vec{v} \times \vec{b}).\\
%    \nabla \cdot \vec{b} &= 0\\
%    \nabla \cdot \vec{v} &= 0.
%\end{split}\end{align}
\begin{align}\begin{split}
    \frac{\partial \vec{u}}{\partial t} + \vec{u} \cdot \nabla \vec{u} &= -\nabla P + \vec{B} \cdot \nabla \vec{B} + \nu \nabla^2 \vec{u} \\
    \frac{\partial \vec{B}}{\partial t} + \vec{u} \cdot \nabla \vec{B} &= \vec{B} \cdot \nabla \vec{u} + \eta \nabla^2 \vec{B} \\
    \nabla \cdot \vec{B} &= 0\\
    \nabla \cdot \vec{u} &= 0.
\end{split}\end{align}
Here $\vec{u}$ is velocity, $\vec{B} = \vec{B}_0 + \vec{b}$ is magnetic field (mean $+$ fluctuations), $P$ is total pressure (thermal $+$ magnetic), $\nu$ is kinematic viscosity, $\eta$ is magnetic diffusivity.
The magnetic field is expressed in units of Alfven speed, $b = B_0/\sqrt{4\pi \rho}$.

The above equation can be compressed further by presenting it in terms of the so-called Els\"asser variables, $\vec{z}^\pm = \vec{u} \pm \vec{b}$.
They represent amplitudes of Alfvenic fluctuations with positive and negative correlations.
Note that no pure wave exists in a turbulent medium, but the interaction can be conveniently written regarding these variables.
Then
\be
    \frac{\partial \vec{z}^\pm}{\partial t} \mp (\vec{B}_0 \cdot \nabla) \vec{z}^\pm + (\vec{z}^\mp \cdot \nabla) \vec{z}^\pm = -\nabla P  + \nu_- \nabla^2 \vec{z}^\pm  + \nu_+ \nabla^2 \vec{z}^\mp,
\ee
where $\nu_\pm \frac{1}{2} (\nu \pm \eta)$.
Importantly, nonlinear interactions are seen to occur as Alfvenic fluctuations of $z^\mp$ via the nonlinear advective derivative $(\vec{z}^\mp \cdot \nabla) \vec{z}^\mp$.

%The first non-linear term $(\vec{B}_0 \cdot \nabla)\vec{z}^\pm \propto (\vec{v}_A \cdot \nabla)\vec{z}^\pm$ is written in k-space as $(k_\parallel v_A) z^\pm$.
%The second non-linear term is, similarly, $(\vec{z}^\mp \cdot \nabla)\vec{z}^\pm$ expressed as $(k_\perp z^\mp) z^\pm$.
%Turbulence is weak when the first term dominates, $k_\parallel v_A \gg k_\perp z^\mp$; 
%it becomes strong when $k_\parallel v_A \sim k_\perp z^\mp$.
%
%Ideal system conserves Els\"asser variables
%\be
%E^\pm = \int (\vec{z}^+)^2 d\vec{x}
%\ee
%that corresponds to conservation of total energy, $E_\mathrm{tot}$ and cross-helicity $H_\mathrm{C}$ as
%\begin{align}\begin{split}
%    E_{\mathrm{tot}} &= \frac{1}{2} \int u^2 + b^2 d\vec{x} \\
%    H_{\mathrm{C}}   &=             \int \vec{u}\cdot \vec{b}  d\vec{x}.
%\end{split}\end{align}


Alfven fluctuations are, therefore, characteristic modes of MHD equations.
This means that all perturbations want to resemble Alfvenic fluctuations.
A typical picture of MHD turbulence is a mutual interaction of (multiple) Alfven wave packets moving with Alfven velocities along the guide field.
The interaction of these wave packets can either be weak (small perturbations of waves) or strong (non-linear quick destroyment of wave coherence during a dynamical time).
It is the Alfvenic part of the MHD perturbations that govern this anisotropic turbulence. 
Hence, the name Alfvenic turbulence.



\section{Weak wave turbulence}

Weak turbulence refers to the picture where wave packets propagate almost freely, and collisions between the waves lead to small perturbations in their structure so that perturbation theory is applicable.
Mathematically, this takes place in the limit of small $\vec{\delta z}$ that represents a perturbation traveling along the $\vec{B}_0$.
The nonlinear term describes the interaction of the perturbations.
Notably, self-interaction of $\vec{\delta z}^-$ or $\vec{\delta z}^+$ is absent.

The dominant nonlinear interaction is a three-wave process.
Dispersion relation and conservation laws for energy and momentum of the waves are, therefore,
\begin{align}\begin{split}
    \omega_n &= k_{\parallel,n} v_A \\
    \pm \omega_1 &= \pm \omega_2 \pm \omega_3 \\
    k_{\parallel,1} &= k_{\parallel,2} + k_{\parallel,3} \\
    k_{\perp,1} &= k_{\perp,2} + k_{\perp,3} 
\end{split}\end{align}
One $\omega_n$ must be zero so let us select $\omega_3=0$ that corresponds to $|k{\parallel,1}| = |k_{\parallel,2}|$.
There is no restriction on $k_{\perp,1,2}$.
The cascade takes place by increasing $k_\perp$ only and preserves the frequencies.

In wave turbulence theory, the interaction strength is
\be
\chi = \frac{k_\perp \delta z}{k_\parallel v_A},
\ee
that describes the ratio of nonlinear shear rate $k_\perp \delta z$ to the wave frequency, $k_\parallel v_A$.
It also serves as an estimate of the nonlinear term to the mean-field term.

In MHD turbulence, the dynamical time scale, $t_\mathrm{dyn} = 1/k_\perp \delta z$ (inverse of shear rate), does not have to be proportional to the cascade time $\tcascd$, as is the case for hydrodynamical turbulence.
Instead, the cascade time is increased by factors of $1/\chi$.

Another way to understand this is to think of wave packet perturbations as a random walk.
Each individual perturbation is $\chi$ in strength, so it takes $(1/\chi)^2$ steps to destroy the coherence of a wavepacket.
Then, the cascade time is
\be
\tcascd = \frac{1}{k_\parallel v_A} \left( \frac{1}{\chi} \right)^2 = \frac{k_\parallel v_A}{(k_\perp \delta z)^2}
\ee

The energy cascade rate, $\varepsilon$, is the energy on each scale divided by the cascade time on that scale.
This rate is expected to be constant through the inertial scale so that
\be
\epsilon = \delta z^2 \frac{(\delta z k_\perp)^2}{v_A k_\parallel}.
\ee
Since the perturbation is weak, $k_\parallel$ is a constant here.
Then the phenomenological cascade rate is determined by $\delta z^2 \sim k_\perp^{-1}$.
This corresponds to a one-dimensional perpendicular spectrum
\be
E(k_\perp) \sim \delta z^2 k_\perp^{-1} \sim k_\perp^{-2}.
\ee
See Zakharov 1992, Galtier et al 2000, 2002.


Such a weak turbulence model is unstable.
Analytic work has demonstrated that MHD turbulence tends to become stronger and not weaker during a cascade (Galtier et al., 2000).
Weak wave turbulence is expected to grow anisotropic with $k_\perp/k_\parallel \sim k_\perp$ since $k_\parallel$ is a constant.
The perturbation strength, $\chi$, is also expected to become stronger at small scales.
This is seen from
\be
\chi = \frac{\delta z k_\perp}{v_A k_\parallel} \sim k_\perp^{1/2} \rightarrow \infty ~ \mathrm{when }~ k_\perp \rightarrow \infty.
\ee



\section{Phenomenological strong MHD turbulence models}\label{sect:theory}

Before we start with phenomenological strong MHD models, let us remark on the actual observations.
Especially the hydrodynamical turbulence is a well-studied problem.
Our best understanding of this system is still roughly a Kolmogorov-like system that predicts $-5/3$ scaling.

Physical and numerical experiments have never agreed with this theoretical scaling.
%It is a solid fact that the scaling is closer to $-1.7$ \red{CHECK} and not $-1.66$ as predicted.
The situation is even worse for MHD cascades since we can not perform detailed physical experiments of plasma turbulence, nor can we simulate large enough MHD systems.%
\footnote{
    The problem is computationally very demanding.
From the simulation perspective, there is roughly one order of magnitude of settling scales below the injection scale $\ell_0$ so that the inertial range begins at scales of $\lesssim l_0/10$.
Similarly, dissipation (and grid-scale) physics (and numerics) are present at roughly one order of magnitude above the dissipation scale $\ell_\nu$ so that a clean inertial range extends to $\gtrsim 10 \ell_\nu$.
To perform scaling measurements with an inertial range of length at least 2 decades, we then need to simulate a turbulent system of a size of at least $(10\times10\times10^2)^3 \sim (10^4)^3$.
Current simulations only start to probe these scales, whereas such a system is only the minimum viable numerical experiment.
}
Therefore, one should not get too carried away by the different theoretical models;
none of them is actually correct.


\subsection{Kolmogorovian theory}\label{sect:K41}

Energy is pumped into a homogenous conducting medium with a fixed rate $\epsilon$.
The dimensionless analysis gives for the energy spectrum \citep{Kolmogorov_1941}
\be
E(k) \sim \epsilon^{2/3} k^{-5/3}.
\ee
The same can be written in terms of average velocity increments
\be
\delta u_\lambda \sim (\epsilon \lambda)^{1/3}.
\ee

\subsection{Iroshnikov-Kraichan theory}\label{sect:K65}

If the $\vec{B}$ field has an important role in energy transfer, then a similar dimensionless analysis gives (Iroshnikov Kraichnan 1965)
\be
E(k) \sim (\epsilon v_A)^{1/2} k^{-3/2}
\ee
and
\be
\delta u_\lambda \sim (\epsilon v_A \lambda)^{1/4},
\ee
for the Alf\'ven speed with density $\rho$
\be
v_A = \frac{B}{\sqrt{4\pi \rho}}.
\ee
The reasoning is based on the fact that Alf\'ven time, $\tau_A \sim 1/k v_A$, is the time at which interactions occur, so the energy must come with a combination $\epsilon v_A$.

The theory is incorrect because it assumes a uniform scale $k$ whereas in reality, in the presence of a strong guide field $\vec{B}_0$, the scales split into $k_{\parallel}$ and $k_{\perp}$.
Therefore, the time scale in the assumption is wrong.


\subsection{Goldreich-Sridhar and critical balance}\label{sect:GS95}

In a strong magnetic field $k_{\parallel} \ll k_{\perp}$.
Parallel direction propagation velocity corresponds to Alf\'ven waves with 
\be
\tau_A = \frac{l_{\parallel}}{v_A},
\ee
whereas perpendicular variation is governed by nonlinear interactions with characteristic times
\be
\tau_{\mathrm{nl}} \sim \frac{l_{\perp}}{\delta u_{\lambda}}.
\ee

For Alf\'venic perturbations $\delta u_{\lambda} \sim \delta b_{\lambda}$.
The two times, $\tau_A$ and $\tau_{\mathrm{nl}}$, are assumed to be equal.
The natural ``cascade time'' must also be of same order, $\tau_c \sim \tau_A \sim \tau_{\mathrm{nl}}$
This gives 
\be
\frac{\delta u_{\lambda}^2}{\tau_c} \sim \epsilon \quad\mathrm{and}\quad
\tau_c \sim \tau_{\mathrm{nl}} \sim \frac{\lambda}{\delta u_{\lambda}},
\ee
so that
\be
\delta u_{\lambda} \sim (\epsilon \lambda)^{1/3}
\ee
and equally \citep{Goldreich_1995, 1997}
\be
E(k_\perp) \sim \epsilon^{2/3} k_{\perp}^{-5/3},
\ee
yielding a Kolmogorov-like scaling for the perpendicular scales.

Simultaneously, along the field, the velocity increment satisfy
\be
\frac{\delta u_{\parallel}^2}{\tau_c} \sim \epsilon \quad\mathrm{and}\quad
\tau_c \sim \tau_A \sim \frac{l_{\parallel}}{v_A}
\ee
so that
\be
\delta u_{l \parallel} \sim \left( \frac{\epsilon l_{\parallel}}{v_A} \right)^{1/2}.
\ee
From here, it follows
\be
l_{\parallel} \sim v_A \epsilon^{-1/3} \lambda^{2/3}.
\ee
Physically $l_{\parallel}$ is the distance an Alfv\'enic pulse travels along the field at speed $v_A$ over time $\tau_{\mathrm{nl}}$.


\subsection{Remark on weak turbulence}


Weak turbulence theory stems from a perturbation in a (assumedly) small ratio $\tau_A/\tau_{\mathrm{nl}}$.
WT scaling originates from
\be
\delta z_{\lambda} \sim \left( \frac{\epsilon}{\tau_A} \right)^{1/4} \lambda^{1/2}
\ee
where $\delta z_{\lambda}$ is perturbed Elsasser field.
This gives a scaling
\be
E(k_{\perp}) \sim \left( \frac{\epsilon}{\tau_A} \right)^{1/2} k_{\perp}^{-2}.
\ee

Eventually, weak turbulence will transition to strong turbulence.
For balanced turbulence, this happens when the perturbation parameter becomes of order unity,
\be
\frac{\tau_A}{\tau_{\mathrm{nl}}} \sim \frac{\tau_A^{3/4} \epsilon^{1/4}}{\lambda^{1/2}} \sim 1
\ee
corresponding to a scale (assuming critical balance)
\be
\lambda_{\mathrm{CB}} \sim \epsilon^{1/2} \tau_A^{3/2}.
\ee


\subsection{Critical balance}

In a strong turbulence, $2D$ structures with $\tau_{\mathrm{nl}} \ll \tau_A$ are unsustainable because of causality:
Information propagates along $\vec{B}$ at $v_A$ so no structure longer than $l_{\parallel} \sim v_A \tau_{\mathrm{nl}}$ can be kept coherent (Boldyrev 2005) \citep{Boldyrev_2005}.


Alfv\'en wave is a basic element of MHD motion.
Because of this, strong magnetic perturbations would ``want'' to resemble Alfv\'ven waves as closely as possible.
Critical balance relates to this:
An Alfv\'enic perturbation decorrelates in roughly one wave period.

\subsection{Dynamical alignement}


An ideal reduced MHD system has two conserved quantities, the energy, $E$, and the cross-helicity, $H^C$.
Assuming energy is dissipated faster than the cross-helicity, we can ask what field configuration minimizes the energy at a given cross-helicity.
This configuration corresponds to a state where the magnetic and velocity fluctuations, $\vec{b}\equiv \delta \vec{B}_\perp$, and $\vec{v}\equiv \delta \vec{V}_\perp$, tend to either align or counter-align.
These are the so-called Alfv\'enic  states, $\vec{b} = \pm \vec{v}$ because they can be described with Els\"asser  variables, $\vec{z}^\pm = \vec{v} \pm \vec{b}$ that are independently exact solutions of the reduced MHD system and physically present an Alfv\'en mode traveling along the dominant guide field.


If a large-scale magnetic field is aligned with the $z$ direction, the resulting turbulent structures are created in the $x-y$ plane.
If there is indeed a tendency for $\vec{v}$ and $\vec{b}$ to align, then locally, the isotropy in the 2D slice perpendicular to the dominant field is broken.
If both quantities, $\vec{v}$ and $\vec{b}$ are aligned, say along the $y$ direction, then the resulting ribbon has a thickness $a$ in the $x$ direction, width $\zeta$ in $y$ direction, and length $l$ in $z$ direction.

Such alignment leads to reduction of the nonlinear interaction strength, $(\vec{z}^\mp \cdot \nabla)\vec{z}^\pm$, because the projection of the two quantities and their gradients, $\vec{b}\cdot\delta\vec{v}_\perp \sim b_y v_x$, is reduced by the anisotropy factor $\sin\theta = a/\zeta$.
In other words, the magnetic field experiences less shearing from the velocity fields.

A subtle, often less clearly communicated detail is what aligns.
Originally, the alignment by Boldyrev was envisioned to take place between the velocity and magnetic fluctuations, $\vec{b} \cdot \nabla \vec{v}$, but later this was shown to be dimensionally incorrect. 
Instead, the alignment is, in reality, thought now to occur between the Els\"asser  variables. %, $\vec{z}^\mp \cdot \nabla \vec{z}^\pm$.

Turbulence can also be hypothesized to form by a tendency of each scale to form structures that act to \textit{minimize} the nonlinear interaction and to equilibrate with a forcing (incoming spectral energy flux that needs to be passed on).
This would be an alternative formulation for the dynamical alignment that relies on the tendency of each scale to form Alfvenic structures to the best of their ability, as each scale is perturbed by the shearing motion of the larger scales. 



Dynamic alignment derives from the same assumption of MHD tendency towards Alfv\'enic nature.
In an Alfv\'en wave, $\vec{u}_\perp$ and $\vec{b}_\perp$ are the same, i.e., plasma flows drag field with them or the field accelerates flows to relax under tension.
However, if the two fields are exactly parallel, there would be no non-linearity.

Boldyrev's theory on dynamical alignment states that the angle, $\theta_{\lambda}$ between the two fields can not be known more precisely than 
\be
\sin \theta_{\lambda} \sim \theta_\lambda \sim \frac{\delta b}{v_A} \ll 1
\ee
Then, the non-linear time is modified to be
\be
\tau_{\mathrm{nl}} \sim \frac{\lambda}{\delta z_\lambda^\mp \sin\theta_\lambda}
\ee
This leads to Kraichnan-type of $-3/2$ scaling
\be
E(k_\perp) \sim (\epsilon v_A)^{1/2} k_\perp^{-3/2}
\ee
and 
\be
l_\parallel \sim v_A^{3/2} \epsilon^{-1/2} \lambda^{1/2}.
\ee

Yet another way to rewrite the scaling relations in terms of the critical balance scale, $\lambda_{\mathrm{CB}}$, is 
\be
E(k_\perp) \sim \epsilon^{2/3} \lambda_{\mathrm{CB}}^{1/6} k_\perp^{-3/2},
\ee
that follows the prediction for the spectrum by Perez et al. 2012, 2014b.
This is sometimes known as aligned turbulence.


\subsection{Intermittency}

Classical turbulence theories rely on the self-similarity of the structures. 
Intermittency means that this assumption is broken, and instead, we introduce all three length scale directions: 
perpendicular $\lambda$, parallel $l_\parallel$, and fluctuation direction $\zeta$.

Intermittency states that eddies are not completely space-filling but have rare fluctuations on top of the ``typical'' ones.
In hydrodynamic turbulence, corrections to K41 theory come in powers of $\lambda/L$.
Similar can be found in MHD turbulence as the self-similarity is broken with the appearance of outer-scale size $L_\parallel$ in the scaling equations.

Mallet \& Schekochihin conjectured that $l_\parallel \sim \lambda^\alpha$, i.e., $l_\parallel/\lambda^\alpha$ has a scale-invariant distribution.
A typical (second) conjecture is that most intense structures are sheets transverse to the local perpendicular direction.

%Note: Some arguments about outer scale structures lead to a prediction on MS17 for alpha=1/2.

\subsection{Phenomenological unified picture of Schekochihin}

In reality, the scaling is expected to be fully $3D$.
From critical balance and the assumption of dynamical alignment, it then follows that that field's spectra depend on:
$-2$ in $l_\parallel$ direction,
$-3/2$ in $\lambda$ direction, and
$-5/3$ in $\zeta$ direction.

In addition, we have
\be
\zeta \sim l_\parallel \frac{\delta z_\lambda}{v_A} \sim l_\parallel \frac{\delta b_\lambda}{v_A}a,
\ee
i.e., $\zeta$ is the typical displacement of a fluid element and a typical perpendicular distance a field line wanders within a structure that is coherent on the parallel scale $l_\parallel$.
Fluctuations must, therefore, preserve coherence in their respective direction, at least on scale $\zeta$.
They are not constrained in the third $\lambda$ direction.
Finally, they are expected to have an angular uncertainty of the order of the angle $\theta_\lambda$ between the two fields.

Finally, we note that this picture is not solid yet.
In contrast, it relies on the refined dynamical alignment conjecture that itself has not proven correct or incorrect.
Another way to think about this conjecture is that it is used to describe intermittency.
Therefore, a complete theory of intermittency is needed for a solid theory of MHD turbulence.


\section{Reconnection-driven cascade disruption}

Turbulent dynamics favors the formation of structures that have a morphology of anisotropic current sheets.
%An anisotropic eddy in this picture is basically an elongated current sheet.
However, current sheets with large aspect ratios are prone to tearing instability.
Therefore, magnetic reconnection resulting from the sheet tearing may alter the turbulent energy cascade.

Magnetic reconnection occurs as a thin, elongated current sheet breaks into magnetic islands.
Notably, this tearing can completely disrupt the energy transfer at a scale $a_{tear}$ where the process occurs.
This means that the turbulent energy transfer would not proceed by passing energy from one scale to another but instead distributing the reconnecting magnetic energy between multiple scales, $<a_{tear}$.

This will result in what I call a \textit{disruption of the turbulent energy cascade}.
Existing work (Loureiro \& Boldyrev 2017; Boldyrev \& Loureiro 2017) considers reconnection as an alternative nonlinear energy transfer mechanism. 
This might, however, not be the case since reconnection does not cleanly pass energy from one scale to the next in the Fourier space.

As an example, Boldyrev \& Loureiro 2017; Mallet et al. 2017 assumed that the spectrum can be calculated by replacing the nonlinear turnover time with the tearing time at all scales smaller than $a_{tear}$.
This gives an energy spectrum $E(k_\perp) \sim k_\perp^{-11/5}$ (assuming a $\tanh$-profile for the current sheets).
Crucially, we need to consider at what scale can this take place?


%\section{Summary}\label{sect:summary}

%\section*{Acknowledgments}
%JN would like to thank XXX stimulating discussions on numerics and plasma simulations. 
%The simulations were performed on resources provided by the Swedish National Infrastructure for Computing (SNIC) at PDC and HPC2N.


\bibliographystyle{aa}
\bibliography{refs}

%\begin{thebibliography}{62}
%\end{thebibliography}

\clearpage
%\onecolumn
\begin{appendix}




\section{More advanced statistical quantities}\label{sect:adv_stat}

\subsection{Structure functions}

Structure functions are averages of differences between turbulent quantities measured at different nearby locations.
General order $p$ structure function of a physical variable $\vec{X}(\vec{x})$ (like $\vec{u}$, $\vec{B}$, etc.) are defined as
\be
S^X_p(r) = \langle [ X(\vec{x}-\vec{r}) - X(\vec{x}) ]^p \rangle,
\ee
where $\langle \ldots \rangle$ denotes averaging (time, volume, ensemble). %; no stance taken on what the exact nature actually is).
This quantity captures the spatial correlation of the field $\vec{X}$.
%For velocity it reduces to kinetic energy, $S^u = 4 E_K$ on a given scale.

Physically structure functions can be thought of as a measure of gradients since Taylor expanding
\be
u(\vec{x} + \vec{r}) - u(\vec{x}) \sim \frac{\partial u}{\partial x} r.
\ee
Especially useful is the second-order structure function, $S_2$, which can be thought of as a measure of energy fluctuations,
\be
S_2^u(r) = \langle [ u(\vec{x}-\vec{r}) - u(\vec{x}) ]^2 \rangle.
\ee
Indeed, spectra and structure functions have a one-to-one correspondence
\be
S_2(\ell) = 2\int_0^\infty \left( 1 - \frac{\sin kr}{kr}\right) E(k) dk
\ee
If spectrum is a power-law, $E(k) \propto k^\alpha$, then by substitution of $k = x/\ell$, we have $S_2(\ell) \propto \ell^{-1-\alpha}$.

From a statistical viewpoint, if the turbulence is self-similar (i.e., has a single-fractal structure), higher-order structure functions are all connected as 
\be
[S_n(r))]^{1/n} \sim [S_m(r))]^{1/m},
\ee
for any arbitrary orders $n$ and $m$.

\subsection{Correlation functions}

Correlation functions are typically used to study temporal behavior.
They can, however, also be generalized to spatial correlations.
Similar to structure functions, we can connect spectra and correlation functions together.

\red{TODO: Autocorrelation}

%try2 on structure functions
The two-point correlation function is
\be
\langle u_i(\vec{r}) u_j(\vec{r'})\rangle = C_{ij}(\vec{r}-\vec{r}') = C_{ij}(\vec{r})
\ee
The isotropic two-point correlation function reduces to
\be
C_{ij}(\vec{r}) = C^{(1)}(r) r_i r_j + C^{(2)}(r) \delta_{ij}.
\ee
Energy and other second-order quantities play an important role in MHD turbulence.
For homogeneous systems, these are defined as
\be
\langle X_i(\vec{k}) Y_j(\vec{k}'\rangle = C_{ij}^{XY}(\vec{k}) [2\pi]^D \delta(\vec{k} + \vec{k}')
\ee

The spectrum and correlation function in real space are connected as
\be
E^X = \frac{1}{2} \langle X^2 \rangle = \frac{1}{2} \int \frac{d\vec{k}}{(2\pi)^D}  C_{ij}^{XY}(\vec{k})
\ee
and
\be
\int E^X(k) dk = \frac{1}{2} \int dk \frac{S_D (D-1)}{(2\pi)^D} k^{D-1} C^{XX}(\vec{k}),
\ee
where $S_D = 2\pi^{D/2} / \Gamma(D/2)$ is the area of $D$-dimensional unit sphere.
Therefore, for isotropic turbulence 
\be
E^X (k) = C^{XX}(\vec{k}) k^{D-1} \frac{S_D (D-1)}{2(2\pi)^D}.
\ee


\section{Deviations from incompressibility and non-relativistic assumptions}\label{sect:kin_theory}

%\red{TODO: deviations from above: compressibility, relativistic motions, }

Astrophysical plasmas that are magnetically dominated and undergo strong non-linear driving can show deviations from the above assumptions.
Most notably, these include relativistic bulk motions that can induce a dynamically important electric field.
For highly conductive flows $E \approx \frac{u}{c} B$.
The electric field remains small for non-relativistic flows, $E \ll B$.
If the driving scale motions are relativistic, the electric field can become large enough to become dynamically important itself.

The same applies to the current, as the plasma can become charge-starved.
From Maxwell's equations, we have that
\be
\nabla \times \vec{B} = \frac{4\pi}{c} \vec{J} + \frac{1}{c} \frac{\partial E}{\partial t}.
\ee
We can ignore the last term for non-relativistic plasma since it is $\propto (u/c)^2$.
Therefore, 
\be
\vec{J} = \frac{c}{4\pi} \nabla\times\vec{B}.
\ee
Hence, both $\vec{E}$ and $\vec{J}$ are dependent variables.
For relativistic flows, current can become charge-starved and dynamically important.

Relativistic bulk motions can also lead to the introduction of compressible modes.
For magnetized plasmas, Alfven speed, $v_A = B/\sqrt{4\pi\rho}$ plays the role of sound speed in the medium.
The fluid is incompressible if $u \ll v_A$.
For relativistic plasma with strong non-linear driving strength, both velocities become comparable, $u \sim v_A$.


Magnetized plasma can support a set of different wave modes.
Most notable is the Alfven wave, which is the incompressible mode.
For Alfven wave $u_\parallel = u_\perp = b_\perp = 0$ but $\delta u_\perp \ne 0$ and $\delta b_\perp \ne 0$.


Cho and Lazarian (2002, 2003) suggest that Alfven mode has an independent cascade, slow mod is passive to Alfven cascade and has the same spectra and isotropy, and fast mode has an independent isotropic acoustic/wave turbulence cascade.
Kowal and Lazarian 2010 report steeper $k^{-2}$ spectra for fast modes.

Realistic compressible turbulence is strongly dependent on the way it is driven;
this defines the initial wave content and degree of compressibility.

Density is primarily perturbed by the slow mode.
In a magnetized plasma with slow mode perturbations, the slow mode is mainly along $\vec{B}_0$.
Beresnyak 2005
Alfven modes only mix these perturbations by shearing motions.

See decaying MHD turbulence that is driven by initial random magnetic fields (e.g., Biskamp 2003) that is qualitatively similar to reconnection turbulence.


Coupling between fluid and magnetic field gives field lines stiffness and elasticity.
This turns unstable perturbations into propagating Alfven waves.
The coupling is not perfect since resistivity allows some slippage of the field across the fluid to take place.
This allows one to tap into the reservoir of free magnetic energy more efficiently.

The driving mechanism is the attractive force of parallel current elements.




\section{Literature List \red{Work-in-progress}}\label{sect:literature}

Objects themselves:
\begin{itemize}
    \item PWN: Woosley 1993
    \item jets from AGNs: Reynolds 1996
    \item GRBs: Wardle 1998
\end{itemize}

%--------------------------------------------------
%plasma turbulence as a generator of non-thermal particles
\subsection{Non-thermal particles from turbulence}
\begin{itemize}
    \item Melrose 1980;
    \item Petrosian 2012;
    \item Lazarian 2012;
\end{itemize}

\subsection{Turbulence in astrophysics}

turbulence in stellar coronae:
Matthaeus 1999, Cranmer 2007

ISM:
Armstrong 1995, Lithwick \& Goldreich 2001

SNRs:
Weiler \& Sramek 1988, Roy 2009

PWN:
Porth 2014, Lyutikov 2019

BH disks:
Balbus \& Hawley 1998, Brandenburg \& Subramanian 2005

jets from AGNs:
Marscher 2008, MacDonald \& Marscher 2018
\citep{MacDonald_2018}

radio lobes:
Vogt \& Ensslin 2005, O'Sullivan 2009

GRBs:
Wardle 1998
Piran 2004, 
\citep{Kumar_2009}

Galaxy clusters:
Zweibel \& Heiles 1997, Subramanian 2006

Laser laboratory plasma:
Sarri 2015


\subsection{Magnetically-dominated turbulence}

Sustained relativistic turbulence (force-free)
\citep{Thompson_1998}: extension of Goldreich \& Sridhar 1995 to exterme relativistic limit (no plasma inertia; force-free MHD).
An anisotropic cascade is formed, and dissipation occurs at the scale of current starvation (when there are not enough charge carriers in plasma to maintain currents from Alf\'en waves).

MHD:
\citep{Cho_2005}
Inoue 2011
\citep{Cho_2014}
\citep{Zrake_2016}

Relativistic MHD:
\citep{Zrake_2012}
\citep{Zrake_2014}

\subsection{Bright non-thermal synchrotron and inverse Compton signatures}
Pulsar magnetospheres and winds:
Buhler \& Blandford 2014

Jets from AGNs: Begelman 1984

Coronae of accretion disks:
\citep{Yuan_2014}


%\subsection{Kinetic turbulence}
%
%\citep{Zhdankin_2017a} Letter\\
%\citep{Zhdankin_2017b} Paper\\
%\citep{Zhdankin_2018} System size convergence\\
%\citep{Zhdankin_2019a} electron-proton plasma\\
%\citep{Zhdankin_2019b} radiative turbulence\\
%\citep{Comisso_2018} acceleration\\
%\citep{Wong_2019} acceleration\\
%\citep{Nattila_2019} Runko and turbulence\\
%\citep{Comisso_2019} acceleration\\
%
%\subsection{Radiative turbulence}
%
%\subsubsection{Analytic work on radiative turbulence}
%\citep{Uzdensky_2018}; \\
%\citep{Zrake_2018} (GRBs) \\
%\citep{Sobacchi_2019}; \\
% 
%\subsubsection{PIC simulations}
%\citep{Zhdankin_2019b} radiative turbulence\\
%
%
%\subsubsection{Fokker-Planck equation in momentum space with radiative cooling term}
%\citep{Schlickeiser_1984, Schlickeiser_1985}; not in original list \citep{Schlickeiser_1989} \\
%\citep{Stawarz_2008} \\


%\subsection{radiative PIC simulations}
%
%\subsubsection{Reconnection}
%\citep{Jaroschek_2009} \\
%\citep{Cerutti_2013, Cerutti_2014, Cerutti_2014a} \\
%\citep{Kagan_2016, Kagan_2016b} \\
%\citep{Hakobyan_2019} \\
%\citep{Werner_2019} \\
%\citep{Schoeffler_2019} \\

%\subsubsection{Decay of magnetostatic equilibria}
%\citep{Yuan_2016} \\
%\citep{Nalewajko_2018} \\
%
%\subsubsection{Pulsar wind}
%\citep{Cerutti_2017} \\
%
%\subsubsection{Pulsar magnetospheres}
%\citep{Cerutti_2016a} \\
%\citep{Philippov_2018} \\
%
%
%\subsubsection{Synchrotron and jitter radiative signatures of collisionless shocks}
%\citep{Medvedev_2009} \\
%\citep{Sironi_2009} \\
%\citep{Kirk_2010} \\
%\citep{Nishikawa_2011} \\
%
%\subsubsection{Radiative turbulence}
%\citep{Zhdankin_2019}


\section{Notes on von Neumann: Recent theories of turbulence}


Turbulence is crucially dimensionality dependent.
Firstly, because solutions of partial differential equations depend on the connection between the equations' term orders and space dimensionality;
NS eqs have the same order as space, and they cross the critical lower bound for regularity when dimensionality changes from $2$ to $3$.
Secondly, in 2D, the vorticity is conserved; it is only created at the boundary and then diffuses into the fluid.
In 3D vortex filaments, strength is conserved, but since the flow may lengthen and convolve the vortex filaments, the actual vorticity in a given volume may increase.
So, a vorticity-increasing mechanism is available in 3D.

Kolmogorov-Onsager-Weizsacker result originates from a remarkable dimensionality analysis of turbulent quantities:
The only quantity that can matter for energy flow is the rate of dissipation $W$ ($=$ energy/mass/time).
Correlations/fluctuations in the flow are captured with $B = \langle u(x+r) - u(x) \rangle$ ($= \cm^2 \second^{-2}$).
Only combination of $r$, $W$, and $B$ is 
\begin{equation}
    B(r) = C W^{2/3} r^{2/}
\end{equation}
Let $F(k)dk$ be the energy contained in the frequency interval from $k$ to $k+dk$.
$F(k)$ is the frequency spectrum of the energy.
Therefore, $B(r)$ and $F(k)$ are Fourier transforms of each other.
Hence, $F(k) \propto k^{-5/3}$ is equivalent to $B(r) \propto r^{2/3}$ (sum of the exponents must be $-1$).
\footnote{
    $F(k) dk = F(r) (\partial k/\partial r) dr$ where $\partial k/\partial r = \partial_r r^{-1} = \ln r$ .
    \red{so???}
}

Specific energy content $E$ (energy per unit mass) and rate of dissipation $W$ (energy per unit mass per time) are obtained from
\begin{equation}
    E = \int F(k) dk
\end{equation}
and
\begin{equation}
    W = \nu \int 2 F(k) k^2 dk
\end{equation}

For Kolmogorov turbulence we obtain $W = 2 \nu A \int_0^{k_s} k^{1/3} dk = \frac{3}{2} \nu A k_s^{4/3}$.

In general, $k$ is the frequency that corresponds to $\ell$ such that 
\begin{equation}
2\pi/k = \ell_k.
\end{equation}
The same is true for energy, 
\begin{equation}
    \langle (u_k)^2 \rangle = A \int_k^\infty k^{-5/3} dk = \frac{3}{2} A k^{-2/3}
\end{equation}
Hence, the typical velocity is
\begin{equation}
    V_k = \sqrt{(u_k)^2} = \sqrt{\frac{3}{2}} \sqrt{A} k^{-1/3}
\end{equation}
Therefore, we can also obtain a Reynolds number at a scale $k$ of
\begin{equation}
    R_k = \frac{L_k V_k}{\nu} = 2\pi (3/2)^{1/2} A^{1/2} \nu^{-1} k^{-4/3}
\end{equation}
Since turbulent motions can not exist for small Reynolds numbers, the cutoff at $k = k_\nu$ coincides with $R_{k_\nu} = 1$

Therefore, from $W/\nu$ we obtain
\begin{equation}
k_s = (W/\nu^3)^{1/4}
\end{equation}
and
\begin{equation}
    A = (3/2) W^{2/3}
\end{equation}
The minimum size of turbulence is, therefore
\begin{equation}
    \ell_{k_\nu} = \ell_0 (R/2\pi)^{-3/4}
\end{equation}



\section{General thoughts on alignment and reconnection-mediated cascades}
On the disruption of an MHD cascade due to magnetic reconnection


\subsection{Dynamically aligned turbulent cascade}

To preserve a power-law scaling form for the spectrum, the cascading eddies should have an (outer-) scale-dependent anisotropy, $\sin\theta \sim (a/L)^{1/4}$.
Since the alignment reduces the nonlinear interaction strength, the cascade becomes necessarily shallower.
For a general intermittency-correction of type $(a/L)^{-\delta}$ with $\delta = 1/4$ we obtain a slope of $p = -3/2$.

For all we care, the cascade could occur with whatever intermittency correction parameter $\delta$, and it would only affect the index of the spectrum.
However, the crucial point to consider is if the alignment can have other effects on the cascade.


\subsection{Onset of tearing instability}


The tearing rate is (assuming $\tanh$-profile for simplicity) $\gamma_{tear} \sim \eta^{1/2} b_a^{1/2} a^{-3/2}$ and it occurs at a wave number $k_{tear} \sim \eta^{1/4} b_a^{-1/4} a^{-5/4}$.
As the scale $a$ decreases, the nonlinear eddy turnover rate increases slower than the tearing rate.
Therefore, at a small scale, the tearing will become important.
The transition scale is $a_{tear}/L \sim S^{-4/7}$ where $S = V_A L/\eta$ is the magnetic Reynolds number (actually Lundquist number to be more accurate).
Kolmogorov-like dissipation scale of turbulence, on the contrary, is $a_\eta/L \sim S^{-3/2}$ (scale at which eddy turnover rate becomes comparable to the dissipation rate $\gamma_\eta \sim \eta/a^2$).

The tearing is estimated to become important at $S \gtrsim 10^6$ (Boldyrev \& Loureiro 2017).
To estimate the scale on which the tearing can become important, we need to know the magnetic diffusion, $\eta$.
The obtained scale $a_{tear}$ should then be compared to the plasma skin depth and the Kolmogorov dissipation scale to determine if the tearing will be important.
\red{What is $a_{tear}$?}

\subsection{Strength of the reconnecting magnetic field component}

The magnitude of the turbulent magnetic fluctuations at scale $a$ can be estimated from the constant energy flux, $b_a^2 (b_a/a)(a/L)^{\delta} \sim \varepsilon$.
The initial energy flux at the outer scale is $\varepsilon = b_0^3 /L$.
So, at a scale of $a$, we have 
%$b_a^3 = (b_0^3/L) (a) (a/L)^{-\delta}$.
$(b_a/b_0)^2 = (a/L)^{2/3 -2\delta/3}$.

For $\delta = 1/4$ we get at the tearing scale $(b_a/b_0)^2 = (a/L)^{1/2} \sim S^{-4/14} \sim 10^{-2}$.
If the initial wave amplitude is, say, $b_0 \sim B_z/100$, then the disruption occurs via a reconnection with ultra-strong guide field, $b/B_z \sim 10^{-6}$.



Eddy turnover rate is $\gamma_{nl} \propto b_a/a^2$.
Energy flux is $\varepsilon \sim b_a^3/a^2 \sim \gamma_a E_a = \mathrm{const}$.
Then the nonlinear eddy turnover rate is $\gamma_{nl} \sim (a/L))^\delta b_a/a \sim \varepsilon^{1/3} L^{-1/6} a^{-1/2}$.


\subsection{Tearing in kinetic Alfv\'en wave turbulence}

Boldyrev claims that tearing resembles MHD when $\eta \rightarrow \gamma d_e^2$, where $d_e$ is the electron skin depth.

Tearing rate is $\gamma_{tear} \sim b_a d_e^2/a^3$.
The inner scale of the mode is at $\delta_{in} \sim d_e$.



\end{appendix}

\end{document}
