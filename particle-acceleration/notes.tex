%\documentclass[referee]{aa}
\documentclass{aa}


\usepackage[utf8]{inputenc}
\usepackage{tabularx}
\usepackage{multirow}
\usepackage{url}
\usepackage{amsmath}	% Advanced maths commands
\usepackage{amssymb}	% Extra maths symbols
\usepackage{graphicx}
\usepackage{color}
\usepackage{natbib}
\usepackage{hyperref}

%\usepackage{newtxtext,newtxmath}
\usepackage{times,txfonts}
\usepackage{bm}

% Depending on your LaTeX fonts installation, you might get better results with one of these:
%\usepackage{mathptmx}
%\usepackage{txfonts}

% Use vector fonts, so it zooms properly in on-screen viewing software
% Don't change these lines unless you know what you are doing
\usepackage[T1]{fontenc}
\usepackage{ae,aecompl}


%landscape figures
%\usepackage{wrapfig}
%\usepackage{lscape}
%\usepackage{rotating}

\bibpunct{(}{)}{;}{a}{}{,} % to follow the A&A style

%Debug addition for collaborators
%\usepackage[switch, modulo]{lineno}
%\linenumbers
%\renewcommand\linenumberfont{\color{red}\normalfont\tiny\sffamily}
%\renewcommand\linenumberfont{\normalfont\tiny\sffamily}

\DeclareUnicodeCharacter{00A0}{ }

\usepackage{listingsutf8}

% configure listings package
\lstset{
         basicstyle=\footnotesize\ttfamily, 
         %numbers=left,               
         numberstyle=\tiny,          
         %stepnumber=2,             
         numbersep=5pt,             
         tabsize=2,                  
         extendedchars=true,         
         breaklines=true,            
         keywordstyle=\color{red},
         frame=b,
 %        keywordstyle=[1]\textbf,   
 %        keywordstyle=[2]\textbf,   
 %        keywordstyle=[3]\textbf,   
 %        keywordstyle=[4]\textbf,   
         %stringstyle=\color{white}\ttfamily, % Farbe der String
         showspaces=false,           
         showtabs=false,            
         xleftmargin=17pt,
         framexleftmargin=17pt,
         framexrightmargin=5pt,
         framexbottommargin=4pt,
         %backgroundcolor=\color{lightgray},
         showstringspaces=false    
}
\lstloadlanguages{C++}
\newcommand{\codeil}[1]{\lstinline!#1!}


\usepackage{amsmath}
\usepackage{amssymb}


%--------------------------------------------------
%Basic commands
\newcommand{\be}{\begin{equation}}
\newcommand{\ee}{\end{equation}}
\newcommand{\bea}{\begin{align}\begin{split}}
\newcommand{\eea}{\end{split}\end{align}}

\newcommand{\fig}[1]{Fig.~\ref{#1}}
\newcommand{\eq}[1]{~(\ref{#1})}
\newcommand{\sect}[1]{Sect.~\ref{#1}}
\newcommand{\appnd}[1]{Appendix \ref{#1}}

\newcommand{\Ten}[2]{\ensuremath{#1 \times 10^{#2}} }


%-------------------------------------------------- 
% Units with proper unit spacing
\newcommand{\unitspace}{\,}

\newcommand{\km}{\ensuremath{\unitspace \mathrm{km}}}
\newcommand{\cm}{\ensuremath{\unitspace \mathrm{cm}}}
\newcommand{\g}{\ensuremath{\unitspace \mathrm{g}}}
\newcommand{\Hz}{\ensuremath{\unitspace \mathrm{Hz}}}
\newcommand{\gcm}{\ensuremath{\unitspace \mathrm{g}\,\mathrm{cm}^{-3}}}
\newcommand{\cms}{\ensuremath{\unitspace \mathrm{cm}\,\mathrm{s}^{-1}}}
\newcommand{\cmss}{\ensuremath{\unitspace \mathrm{cm}\,\mathrm{s}^{-2}}}
\newcommand{\dyncm}{\ensuremath{\unitspace \mathrm{dyn}\,\mathrm{cm}^{-2}}}
\newcommand{\erg}{\ensuremath{\unitspace \mathrm{erg}}}
\newcommand{\ergs}{\ensuremath{\unitspace \mathrm{erg}\,\mathrm{s}^{-1}}}

\newcommand{\Gauss}{\ensuremath{\unitspace \mathrm{G}}}
\newcommand{\Kelvin}{\ensuremath{\unitspace \mathrm{K}}}

%\newcommand{\Msun}{\ensuremath{\unitspace M_{\odot}}}
\newcommand{\Msun}{\ensuremath{\unitspace \mathrm{M}_{\odot}}}


%--------------------------------------------------
% colors
\newcommand{\red}[1]{\textcolor{red}{#1}}
\newcommand{\green}[1]{\textcolor{green}{#1}}
\newcommand{\blue}[1]{\textcolor{blue}{#1}}



%--------------------------------------------------
% vectors

%normal 3-vectors
%\renewcommand{\vec}[1]{\ensuremath{\mathbf{#1}}}
%\renewcommand{\vec}[1]{\ensuremath{\boldsymbol{#1}}}
\renewcommand{\vec}[1]{\ensuremath{\bm{\mathit{#1}}}}

%four-vectors
\makeatletter
\def\fvec#1{\underline{\sbox\tw@{$#1$}\dp\tw@\z@\box\tw@}}
\makeatother


%--------------------------------------------------
% math

% integrals
\newcommand{\ud}{\mathrm{d}} % differential d

% derivatives
\newcommand{\pD}[2]{\ensuremath{\partial_{#1}}} %partial derivative
\newcommand{\fracpD}[2]{\ensuremath{\frac{\partial {#1} }{\partial {#2} }}} %frac partial

\newcommand{\derfrac}[2]{\frac{\ud #1}{\ud #2}} % fractional deriv
\newcommand{\parfrac}[2]{\frac{\partial #1}{\partial #2}} %fractional partial deriv
\newcommand{\parD}[1]{\partial_{#1}} %partial derivative shorthand

%misc
\newcommand{\osim}{\mathord{\sim}} % ordo O

% statistics
\newcommand{\bra}[1]{\ensuremath{\langle #1 \rangle}}
\newcommand{\braket}[2]{\ensuremath{ \langle #1 ~|~ #2 \rangle}}


%--------------------------------------------------
% programming

% writing c++ is highly non-trivial
%\newcommand{\cpp}{C\nolinebreak\hspace{-.05em}\raisebox{.4ex}{\tiny\bf +}\nolinebreak\hspace{-.10em}\raisebox{.4ex}{\tiny\bf +{rbrace}{rbrace}}}
%\newcommand{\cpp}{C\nolinebreak\hspace{-.05em}\raisebox{.4ex}{\tiny\bf +}\nolinebreak\hspace{-.10em}\raisebox{.4ex}{\tiny\bf +}}
%\def\cpp{{C\nolinebreak[4]\hspace{-.05em}\raisebox{.4ex}{\tiny\bf ++}}}
%\def\CC{lbrace}{lbrace}C\nolinebreak[4]\hspace{-.05em}\raisebox{.4ex}{\tiny\bf ++{rbrace}{rbrace}{rbrace}
%\usepackage{relsize}
%\newcommand\cpp{C\nolinebreak\hspace{-.05em}\raisebox{.4ex}{\relsize{-3}{\textbf{+}}}\nolinebreak\hspace{-.10em}\raisebox{.4ex}{\relsize{-3}{\textbf{+}}}}
%\newcommand\cpp{C\nolinebreak[4]\hspace{-.05em}\raisebox{.4ex}{\relsize{-3}{\textbf{++}}}}

\newcommand{\cpp}{C\texttt{++}}
\newcommand{\cppv}[1]{C\texttt{++}$#1$}

\newcommand{\python}{\textsc{Python}}





\newcommand{\Cfl}{\hat{c}}
\newcommand{\hf}{\frac{1}{2}}

\newcommand{\Bhat}{\vec{\hat{B}}}
\newcommand{\Ehat}{\vec{\hat{E}}}
\newcommand{\Jhat}{\vec{\hat{J}}}
\newcommand{\uhat}{\vec{\hat{u}}}
\newcommand{\vhat}{\vec{\hat{v}}}
\newcommand{\xhat}{\vec{\hat{x}}}
\newcommand{\that}{\hat{t}}

\newcommand{\gammam}{\langle \gamma \rangle}

\newcommand{\Nppc}{N_{\mathrm{ppc}}}

\newcommand{\f}[1]{\ensuremath{f_{\mathrm{ {#1} }}}}
\newcommand{\dt}{\ensuremath{\Delta t}} %delta t
\newcommand{\dd}{\mathrm{d}}


%spesific additions
\newcommand{\runko}{\textsc{Runko}}

\newcommand{\corgi}{\textsc{corgi}}
%\newcommand{\python}{\textsc{Python3}}
\newcommand{\tristan}{\textsc{Tristan}}
\newcommand{\tristanmp}{\textsc{Tristan-MP}}
\newcommand{\pybind}{\textsc{Pybind11}}
\newcommand{\eigen}{\textsc{Eigen}}
\newcommand{\numpy}{\textsc{Numpy}}
\newcommand{\scipy}{\textsc{Scipy}}
\newcommand{\matplotlib}{\textsc{Matplotlib}}


\newcommand{\todo}[1]{\red{#1}}


%________________________________________________________________
\begin{document}


\title{Particle acceleration}

\author{J.\,N\"attil\"a\inst{1}}

\institute{Nordita, KTH Royal Institute of Technology and Stockholm University, Roslagstullsbacken 23, SE-10691 Stockholm, Sweden. \email{joonas.nattila@su.se}
}

\date{Received XXX / Accepted XXX}


\abstract{
    Blaa blaa.
}

\keywords{ Plasmas -- Turbulence -- Methods: numerical }


\maketitle

%________________________________________________________________
%\section{Introduction}\label{sec:intro}
\citep{birdsall1985}

\section{Theory}\label{sect:theory}

\subsection{Main particle acceleration questions}

\begin{enumerate}
    \item How efficient is the acceleration?
    \item What is the slope of the power-law high-energy tail?
    \item What is the maximum attainable energy?
    \item Which physicala mechanisms govern the injection of particles from thermal pool to higher energies?
    \item What timescales are particles accelerated?
\end{enumerate}

Fully-kinetic simulations allow to study the acceleration process and the possible subsequent back-reaction self-consistently.
Albeit computationally heavy, very useful.


\citep{Fermi_1949}
\citep{Fermi_1954}

\citep{Lemoine_2019}

Fermi scheme can apply to any flow for which to electromagnetic $4-$scalers fulfil $\vec{E} \cdot \vec{B} = 0$ and $E^2 - B^2 < 0$.
Under these conditions one can always boost to a frame for which $\vec{E}$ vanishes, by
\be
\vec{\beta}_{\vec{B}} = \frac{\vec{E} \times \vec{B}}{B^2}.
\ee
It is important to note here that this only occurs when ideal magnetohydrodynamics does not apply, since in that case the aforementioned velocity coincides with the local plasma bulk velocity.


Mean $4$-momentum change $\langle \Delta p^\alpha / \Delta t \langle$.
Diffusion tensor $\langle \Delta p^\alpha \Delta p^\beta / \Delta t \langle$ for $\alpha,\beta = 0,\ldots,4$.

\subsection{Fermi-like diffusion}
Based on Pelletiere 2017

relative velocity $\beta_{rel}$ between upstream and downstream flows (velocity of the downstream plasma relative to the upstream frame).
Pitch angle cosine $\mu_1$ before scattering and scattered pitch angle cosine $\mu_2$ when it is coming back (expressed in the upstream rest-frame).
Its energy gain is 
\be
G = \frac{1-\beta_{rel} \mu_1}{1-\beta_{rel} \mu_2}
\ee
Choosing downstream reference frame, $\beta_{rel} \approx 1-1/\gamma_{sh}^2$.
Pitch angle cosine of the crossing particle has $-1 < \mu_1 < \beta_{sh}$; 
A particle coming back from downstream to upstream has $\beta_{sh} < \mu_2 < 1$.

Average energy gain (assuming incorrectly) isotropic distribution functions and independent $\mu_1$ and $\mu_2$:
\be
\langle G \rangle = \left( 1 + \frac{2}{3} | \beta_{rel} | \right)^2.
\ee
Moreover, because the almost isotropic distirbution, probability of escapre through advection in the downstream plasma is $ P_{\mathrm{esc}} = 4 | \beta_d | $.
The index of the powerlaw distribution $dN/dp \propto p^{-s}$ is then Bell 1978
\be
s = 1 - \frac{\ln (1 - P_{esc})}{\ln \langle G \rangle} 
    \approx 1 + 3 |\frac{\beta_d}{\beta_{rel}}|
\ee
i.e., $s=2$ for strong adiabatic shocks where $|\beta_{rel} = 3 | \beta_d |$.
In reality, the strong anisotropy of the distribution function complicates these calculations.

Note that this means that escapre probability $P_{esc}$ controls the non-thermal slope.

Fermi acceleration is possible if scattering frequency in the turbulence exceeds gyrofrequency in the background field.
Turbulent scattering frequency
\be
\nu = \frac{e^2 \delta B_{|d}^2 \tau_{c|d}}{m^2 c^2 \gamma^2}.
\ee
This means that $\nu > \omega_p = e B_{|d}/m c \gamma$.
Alternatively,
\be
\sqrt{\sigma} < \zeta_B
\ee

Here $\zeta_B$ is the fraction of energy going to the turbulence energy as
\be
W_{em} = \zeta_B \gamma_{sh}^2 n_u mc^2
\ee




\subsection{Diffusive shock acceleration (DSA)}
Most widely accepted theory for non-thermal particle population generation 
Blandford \& Eichler 1987
DSA ssumes presence of turbulence in the upstream.
Energetics particles scattered by MHD waves gain energy by diffusively crossing the shock front back and forth many times.
Central problem is the injection since DSA is only effective for energetic particles that can interact with MHD waves.


\subsection{Shock-surfing acceleration}
Thought to be a pre-acceleration mechanism that can feed particles to the later-stage diffusive processes.

Electrons can be accelerated by an interaction with electrostatic waves from Buneman instability.
Shimada \& Hoshino 2000
Hoshino \& Shimada 2002

In turbulent ramp-overshoot region electrons undergo stochastic Fermi-like acceleration (Bohdan 2017) or stochastic shock drift acceleration in the quasi-perpendicular case (Matsumoto 2017)

Sound speed
\be
c_s^2 = \frac{\Gamma_{\mathrm{AD}} k_{\mathrm{B}} T_i}{m_i}
\ee
and $k_{\mathrm{B}} T_i = \frac{1}{2} m_i v_{\mathrm{th},i}^2$ where $v_{\mathrm{th},i}$ is the most probable speed.






\subsection{Shock mediators}
low-magnetization: Weibel
splits current and subsequently also charge into filaments

high-magnetization: magnetic reflection
magnetic reflection of the compressed downstream magnetic field.
Initially everything is moving following $\vec{E} \times \vec{B}$ motion.
Subsequently, a reflected particle sees a wrong sign of $\vec{E}$ field  and undergo Larmor gyration.
This gyration causes positive and negative particles to go in opposite directions transverse to the flow and the associated transverse current increases the magnetic field.
Incoming particles now see a jump in magnetic field and undergo gyration that deccelerates and so also increases the local density of the flow.
In magnetized case, the transition therefore happens within a few Larmor radii in the compressed field.


%\section{Results}\label{sect:results}

%\section{Summary}\label{sect:summary}

%\section*{Acknowledgments}
%JN would like to thank XXX stimulating discussions on numerics and plasma simulations. 
%The simulations were performed on resources provided by the Swedish National Infrastructure for Computing (SNIC) at PDC and HPC2N.


\bibliographystyle{aa}
\bibliography{refs}

%\begin{thebibliography}{62}
%\end{thebibliography}

%\clearpage
%\onecolumn
%\begin{appendix}
%\end{appendix}

\end{document}
