%\documentclass[referee]{aa}
\documentclass{aa}

%\documentclass[usenatbib,onecolumn]{mnras}
%\documentclass[usenatbib,twocolumn]{mnras}
%\pdfoutput=1
%\pdfminorversion=5

\usepackage[utf8]{inputenc}
\usepackage{tabularx}
\usepackage{multirow}
\usepackage{url}
\usepackage{amsmath}	% Advanced maths commands
\usepackage{amssymb}	% Extra maths symbols
\usepackage{graphicx}
\usepackage{color}
\usepackage{natbib}
\usepackage{hyperref}

\usepackage[T1]{fontenc}
\usepackage{ae,aecompl}
\usepackage{bm}

%landscape figures
%\usepackage{wrapfig}
%\usepackage{lscape}
%\usepackage{rotating}

\usepackage{amsmath}
\usepackage{amssymb}


%--------------------------------------------------
%Basic commands
\newcommand{\be}{\begin{equation}}
\newcommand{\ee}{\end{equation}}
\newcommand{\bea}{\begin{align}\begin{split}}
\newcommand{\eea}{\end{split}\end{align}}

\newcommand{\fig}[1]{Fig.~\ref{#1}}
\newcommand{\eq}[1]{~(\ref{#1})}
\newcommand{\sect}[1]{Sect.~\ref{#1}}
\newcommand{\appnd}[1]{Appendix \ref{#1}}

\newcommand{\Ten}[2]{\ensuremath{#1 \times 10^{#2}} }


%-------------------------------------------------- 
% Units with proper unit spacing
\newcommand{\unitspace}{\,}

\newcommand{\km}{\ensuremath{\unitspace \mathrm{km}}}
\newcommand{\cm}{\ensuremath{\unitspace \mathrm{cm}}}
\newcommand{\g}{\ensuremath{\unitspace \mathrm{g}}}
\newcommand{\Hz}{\ensuremath{\unitspace \mathrm{Hz}}}
\newcommand{\gcm}{\ensuremath{\unitspace \mathrm{g}\,\mathrm{cm}^{-3}}}
\newcommand{\cms}{\ensuremath{\unitspace \mathrm{cm}\,\mathrm{s}^{-1}}}
\newcommand{\cmss}{\ensuremath{\unitspace \mathrm{cm}\,\mathrm{s}^{-2}}}
\newcommand{\dyncm}{\ensuremath{\unitspace \mathrm{dyn}\,\mathrm{cm}^{-2}}}
\newcommand{\erg}{\ensuremath{\unitspace \mathrm{erg}}}
\newcommand{\ergs}{\ensuremath{\unitspace \mathrm{erg}\,\mathrm{s}^{-1}}}

\newcommand{\Gauss}{\ensuremath{\unitspace \mathrm{G}}}
\newcommand{\Kelvin}{\ensuremath{\unitspace \mathrm{K}}}

%\newcommand{\Msun}{\ensuremath{\unitspace M_{\odot}}}
\newcommand{\Msun}{\ensuremath{\unitspace \mathrm{M}_{\odot}}}


%--------------------------------------------------
% colors
\newcommand{\red}[1]{\textcolor{red}{#1}}
\newcommand{\green}[1]{\textcolor{green}{#1}}
\newcommand{\blue}[1]{\textcolor{blue}{#1}}



%--------------------------------------------------
% vectors

%normal 3-vectors
%\renewcommand{\vec}[1]{\ensuremath{\mathbf{#1}}}
%\renewcommand{\vec}[1]{\ensuremath{\boldsymbol{#1}}}
\renewcommand{\vec}[1]{\ensuremath{\bm{\mathit{#1}}}}

%four-vectors
\makeatletter
\def\fvec#1{\underline{\sbox\tw@{$#1$}\dp\tw@\z@\box\tw@}}
\makeatother


%--------------------------------------------------
% math

% integrals
\newcommand{\ud}{\mathrm{d}} % differential d

% derivatives
\newcommand{\pD}[2]{\ensuremath{\partial_{#1}}} %partial derivative
\newcommand{\fracpD}[2]{\ensuremath{\frac{\partial {#1} }{\partial {#2} }}} %frac partial

\newcommand{\derfrac}[2]{\frac{\ud #1}{\ud #2}} % fractional deriv
\newcommand{\parfrac}[2]{\frac{\partial #1}{\partial #2}} %fractional partial deriv
\newcommand{\parD}[1]{\partial_{#1}} %partial derivative shorthand

%misc
\newcommand{\osim}{\mathord{\sim}} % ordo O

% statistics
\newcommand{\bra}[1]{\ensuremath{\langle #1 \rangle}}
\newcommand{\braket}[2]{\ensuremath{ \langle #1 ~|~ #2 \rangle}}


%--------------------------------------------------
% programming

% writing c++ is highly non-trivial
%\newcommand{\cpp}{C\nolinebreak\hspace{-.05em}\raisebox{.4ex}{\tiny\bf +}\nolinebreak\hspace{-.10em}\raisebox{.4ex}{\tiny\bf +{rbrace}{rbrace}}}
%\newcommand{\cpp}{C\nolinebreak\hspace{-.05em}\raisebox{.4ex}{\tiny\bf +}\nolinebreak\hspace{-.10em}\raisebox{.4ex}{\tiny\bf +}}
%\def\cpp{{C\nolinebreak[4]\hspace{-.05em}\raisebox{.4ex}{\tiny\bf ++}}}
%\def\CC{lbrace}{lbrace}C\nolinebreak[4]\hspace{-.05em}\raisebox{.4ex}{\tiny\bf ++{rbrace}{rbrace}{rbrace}
%\usepackage{relsize}
%\newcommand\cpp{C\nolinebreak\hspace{-.05em}\raisebox{.4ex}{\relsize{-3}{\textbf{+}}}\nolinebreak\hspace{-.10em}\raisebox{.4ex}{\relsize{-3}{\textbf{+}}}}
%\newcommand\cpp{C\nolinebreak[4]\hspace{-.05em}\raisebox{.4ex}{\relsize{-3}{\textbf{++}}}}

\newcommand{\cpp}{C\texttt{++}}
\newcommand{\cppv}[1]{C\texttt{++}$#1$}

\newcommand{\python}{\textsc{Python}}





%Debug addition for collaborators
%\usepackage[switch, modulo]{lineno}
%\linenumbers
%\renewcommand\linenumberfont{\color{red}\normalfont\tiny\sffamily}
%\renewcommand\linenumberfont{\normalfont\tiny\sffamily}

\DeclareUnicodeCharacter{00A0}{ }

%\newcommand{\red}[1]{\textcolor{red}{#1}}
%\newcommand{\blue}[1]{\textcolor{blue}{#1}}
%\newcommand{\green}[1]{\textcolor{green}{#1}}
%\newcommand{\Msun}{\ensuremath{\mathrm{M}_{\sun}}}
%\newcommand{\D}[2]{\ensuremath{\frac{\partial {#1} }{\partial {#2} }}}


%normal 3-vectors
%\renewcommand{\vec}[1]{\ensuremath{\mathbf{#1}}}
%\renewcommand{\vec}[1]{\ensuremath{\boldsymbol{#1}}}

%four-vectors
%\makeatletter
%\def\fvec#1{\underline{\sbox\tw@{$#1$}\dp\tw@\z@\box\tw@}}
%\makeatother

% configure listings package
\usepackage{listingsutf8}
\lstset{
         basicstyle=\footnotesize\ttfamily, 
         %numbers=left,               
         numberstyle=\tiny,          
         %stepnumber=2,             
         numbersep=5pt,             
         tabsize=2,                  
         extendedchars=true,         
         breaklines=true,            
         keywordstyle=\color{red},
         frame=b,
 %        keywordstyle=[1]\textbf,   
 %        keywordstyle=[2]\textbf,   
 %        keywordstyle=[3]\textbf,   
 %        keywordstyle=[4]\textbf,   
         %stringstyle=\color{white}\ttfamily, % Farbe der String
         showspaces=false,           
         showtabs=false,            
         xleftmargin=17pt,
         framexleftmargin=17pt,
         framexrightmargin=5pt,
         framexbottommargin=4pt,
         %backgroundcolor=\color{lightgray},
         showstringspaces=false    
}

\newcommand{\dt}{\ensuremath{\Delta t}}


%________________________________________________________________
\begin{document}

%\title{Identifying current sheet structures in magnetically-dominated turbulent collisionless plasmas}
\title{Force-free magnetohydrodynamics}
%\titlerunning{Detecting current sheet structures}

\author{J.\,N\"attil\"a\inst{1}}
\institute{Nordita, KTH Royal Institute of Technology and Stockholm University, Roslagstullsbacken 23, SE-10691 Stockholm, Sweden \email{joonas.nattila@su.se}
}

\date{Received XXX / Accepted XXX}

\abstract{
    Force-free MHD equations.
}


\keywords{turbulence --- plasmas --- non-thermal particles}

\maketitle

%________________________________________________________________
\section{Theory}\label{sec:theory}

Blandford gives force-free equation set as Maxwell's equations
\be
\frac{1}{c} \frac{\partial \vec{E}}{\partial t} = \nabla \times \vec{B}
- \frac{4\pi}{c} \vec{J}
\ee
and
\be
\frac{1}{c} \frac{\partial \vec{B}}{\partial t} = -\nabla \times \vec{E}
\ee
in addition to current term
\be
\vec{J} = \frac{c}{4\pi} (\nabla \cdot \vec{E}) \frac{\vec{E} \times \vec{B}}{B^2} 
+ \frac{c}{4\pi} \vec{B} \frac{\vec{B} \cdot \nabla \times \vec{B} - \vec{E} \cdot \nabla \times \vec{E}}{B^2}
\ee


The current can be represented as
\be
\vec{J} = \vec{J}_{\parallel} + \vec{J}_{\mathrm{drift}},
\ee
Here the first term  
\be
\vec{J}_{\parallel} = \vec{B} \frac{\vec{B} \cdot \nabla \times \vec{B} - \vec{E} \cdot \nabla \times \vec{E} }{B^2} 
\ee
is the parallel conduction current that maintains the force-free condition $\vec{E} \cdot \vec{B} = 0$ reducing to $(\nabla \times \vec{B})_{\parallel}$ in steady state.
The second term
\be
\vec{J}_{\mathrm{drift}} = \nabla \cdot \vec{E} \frac{\vec{E} \times \vec{B}}{B^2}
\ee
is the drift current given by the $\vec{E} \times \vec{B}$ advection of the charge density.

Effect of parallel current can be mimicked with an even simpler system by reseting values of $\vec{E}$ such that $\vec{E}_{\parallel}=0$.
This is equivalent of replacing the parallel current with a resistive term $\sigma_{\parallel} E_{\parallel}$ where $\sigma_{\parallel} \rightarrow \infty$.
Note that retention of the parallel current results in much lower diffusion error.

Force-free evolution can lead to configurations in which $B^2 - E^2 > 0$ is violated.
Therefore, we set
\be
\vec{E} \rightarrow \sqrt{ \frac{B^2}{E^2} } \vec{E}.
\ee
This acts like a small, highly localized source of dissipation in current sheets.
Other simpler option is to set $E \leq B$ after every time step by reseting the electric field to $E = B$ in the regions where $E > B$.

Physically these limiters can be thought of converting electromagnetic energy to thermal energy and radiation.
This violation can happen during interaction of strong waves as the Alf\'en speed becomes imaginary and leads to exponential instability, which physically would be averted by dissipation. 
If the $\vec{E} \times \vec{B}$ drift approximation is violated, particles will be accelerated while decoupling form the magnetic field lines. 
This will restore the $E=B$ on plasma timescales (without taking any stance on the actual microphysics).
This can be parameterized by tracking $J_{\perp} = \sigma_{\perp} E$ when $E>B$ and $J_{\perp} = 0$ otherwise where the perpendicular conductivity $\sigma_{\perp} \rightarrow \infty$.

Another example is the formation of current sheets.
In this case the magnetic dominance is invalid (magnetic field reverses and goes through zero) and the plasma pressure must sustain the sheet.
However, in FFE pressure is not included so we try to keep the current sheets unresolved on the grid with a jump over one cell.



Third-order Runge-Kutta time integration.
We maintain time alignment of the fields.
Linear interpolation of the fields to the same location on the grid.



%\section*{Acknowledgments}
%SNIC.

%\bibliographystyle{aa}
%\bibliography{refs}

%\begin{thebibliography}{62}
%\end{thebibliography}


%\clearpage
%\begin{appendix}
%\section{Appendix}
%
%\end{appendix}



\end{document}
