%\documentclass[referee]{aa}
\documentclass{aa}


\usepackage[utf8]{inputenc}
\usepackage{tabularx}
\usepackage{multirow}
\usepackage{url}
\usepackage{amsmath}	% Advanced maths commands
\usepackage{amssymb}	% Extra maths symbols
\usepackage{graphicx}
\usepackage{color}
\usepackage{natbib}
\usepackage{hyperref}

%\usepackage{newtxtext,newtxmath}
\usepackage{times,txfonts}
\usepackage{bm}

% Depending on your LaTeX fonts installation, you might get better results with one of these:
%\usepackage{mathptmx}
%\usepackage{txfonts}

% Use vector fonts, so it zooms properly in on-screen viewing software
% Don't change these lines unless you know what you are doing
\usepackage[T1]{fontenc}
\usepackage{ae,aecompl}


%landscape figures
%\usepackage{wrapfig}
%\usepackage{lscape}
%\usepackage{rotating}

\bibpunct{(}{)}{;}{a}{}{,} % to follow the A&A style

%Debug addition for collaborators
%\usepackage[switch, modulo]{lineno}
%\linenumbers
%\renewcommand\linenumberfont{\color{red}\normalfont\tiny\sffamily}
%\renewcommand\linenumberfont{\normalfont\tiny\sffamily}

\DeclareUnicodeCharacter{00A0}{ }

\usepackage{listingsutf8}

% configure listings package
\lstset{
         basicstyle=\footnotesize\ttfamily, 
         %numbers=left,               
         numberstyle=\tiny,          
         %stepnumber=2,             
         numbersep=5pt,             
         tabsize=2,                  
         extendedchars=true,         
         breaklines=true,            
         keywordstyle=\color{red},
         frame=b,
 %        keywordstyle=[1]\textbf,   
 %        keywordstyle=[2]\textbf,   
 %        keywordstyle=[3]\textbf,   
 %        keywordstyle=[4]\textbf,   
         %stringstyle=\color{white}\ttfamily, % Farbe der String
         showspaces=false,           
         showtabs=false,            
         xleftmargin=17pt,
         framexleftmargin=17pt,
         framexrightmargin=5pt,
         framexbottommargin=4pt,
         %backgroundcolor=\color{lightgray},
         showstringspaces=false    
}
\lstloadlanguages{C++}
\newcommand{\codeil}[1]{\lstinline!#1!}


\usepackage{amsmath}
\usepackage{amssymb}


%--------------------------------------------------
%Basic commands
\newcommand{\be}{\begin{equation}}
\newcommand{\ee}{\end{equation}}
\newcommand{\bea}{\begin{align}\begin{split}}
\newcommand{\eea}{\end{split}\end{align}}

\newcommand{\fig}[1]{Fig.~\ref{#1}}
\newcommand{\eq}[1]{~(\ref{#1})}
\newcommand{\sect}[1]{Sect.~\ref{#1}}
\newcommand{\appnd}[1]{Appendix \ref{#1}}

\newcommand{\Ten}[2]{\ensuremath{#1 \times 10^{#2}} }


%-------------------------------------------------- 
% Units with proper unit spacing
\newcommand{\unitspace}{\,}

\newcommand{\km}{\ensuremath{\unitspace \mathrm{km}}}
\newcommand{\cm}{\ensuremath{\unitspace \mathrm{cm}}}
\newcommand{\g}{\ensuremath{\unitspace \mathrm{g}}}
\newcommand{\Hz}{\ensuremath{\unitspace \mathrm{Hz}}}
\newcommand{\gcm}{\ensuremath{\unitspace \mathrm{g}\,\mathrm{cm}^{-3}}}
\newcommand{\cms}{\ensuremath{\unitspace \mathrm{cm}\,\mathrm{s}^{-1}}}
\newcommand{\cmss}{\ensuremath{\unitspace \mathrm{cm}\,\mathrm{s}^{-2}}}
\newcommand{\dyncm}{\ensuremath{\unitspace \mathrm{dyn}\,\mathrm{cm}^{-2}}}
\newcommand{\erg}{\ensuremath{\unitspace \mathrm{erg}}}
\newcommand{\ergs}{\ensuremath{\unitspace \mathrm{erg}\,\mathrm{s}^{-1}}}

\newcommand{\Gauss}{\ensuremath{\unitspace \mathrm{G}}}
\newcommand{\Kelvin}{\ensuremath{\unitspace \mathrm{K}}}

%\newcommand{\Msun}{\ensuremath{\unitspace M_{\odot}}}
\newcommand{\Msun}{\ensuremath{\unitspace \mathrm{M}_{\odot}}}


%--------------------------------------------------
% colors
\newcommand{\red}[1]{\textcolor{red}{#1}}
\newcommand{\green}[1]{\textcolor{green}{#1}}
\newcommand{\blue}[1]{\textcolor{blue}{#1}}



%--------------------------------------------------
% vectors

%normal 3-vectors
%\renewcommand{\vec}[1]{\ensuremath{\mathbf{#1}}}
%\renewcommand{\vec}[1]{\ensuremath{\boldsymbol{#1}}}
\renewcommand{\vec}[1]{\ensuremath{\bm{\mathit{#1}}}}

%four-vectors
\makeatletter
\def\fvec#1{\underline{\sbox\tw@{$#1$}\dp\tw@\z@\box\tw@}}
\makeatother


%--------------------------------------------------
% math

% integrals
\newcommand{\ud}{\mathrm{d}} % differential d

% derivatives
\newcommand{\pD}[2]{\ensuremath{\partial_{#1}}} %partial derivative
\newcommand{\fracpD}[2]{\ensuremath{\frac{\partial {#1} }{\partial {#2} }}} %frac partial

\newcommand{\derfrac}[2]{\frac{\ud #1}{\ud #2}} % fractional deriv
\newcommand{\parfrac}[2]{\frac{\partial #1}{\partial #2}} %fractional partial deriv
\newcommand{\parD}[1]{\partial_{#1}} %partial derivative shorthand

%misc
\newcommand{\osim}{\mathord{\sim}} % ordo O

% statistics
\newcommand{\bra}[1]{\ensuremath{\langle #1 \rangle}}
\newcommand{\braket}[2]{\ensuremath{ \langle #1 ~|~ #2 \rangle}}


%--------------------------------------------------
% programming

% writing c++ is highly non-trivial
%\newcommand{\cpp}{C\nolinebreak\hspace{-.05em}\raisebox{.4ex}{\tiny\bf +}\nolinebreak\hspace{-.10em}\raisebox{.4ex}{\tiny\bf +{rbrace}{rbrace}}}
%\newcommand{\cpp}{C\nolinebreak\hspace{-.05em}\raisebox{.4ex}{\tiny\bf +}\nolinebreak\hspace{-.10em}\raisebox{.4ex}{\tiny\bf +}}
%\def\cpp{{C\nolinebreak[4]\hspace{-.05em}\raisebox{.4ex}{\tiny\bf ++}}}
%\def\CC{lbrace}{lbrace}C\nolinebreak[4]\hspace{-.05em}\raisebox{.4ex}{\tiny\bf ++{rbrace}{rbrace}{rbrace}
%\usepackage{relsize}
%\newcommand\cpp{C\nolinebreak\hspace{-.05em}\raisebox{.4ex}{\relsize{-3}{\textbf{+}}}\nolinebreak\hspace{-.10em}\raisebox{.4ex}{\relsize{-3}{\textbf{+}}}}
%\newcommand\cpp{C\nolinebreak[4]\hspace{-.05em}\raisebox{.4ex}{\relsize{-3}{\textbf{++}}}}

\newcommand{\cpp}{C\texttt{++}}
\newcommand{\cppv}[1]{C\texttt{++}$#1$}

\newcommand{\python}{\textsc{Python}}





\newcommand{\Cfl}{\hat{c}}
\newcommand{\hf}{\frac{1}{2}}

\newcommand{\Bhat}{\vec{\hat{B}}}
\newcommand{\Ehat}{\vec{\hat{E}}}
\newcommand{\Jhat}{\vec{\hat{J}}}
\newcommand{\uhat}{\vec{\hat{u}}}
\newcommand{\vhat}{\vec{\hat{v}}}
\newcommand{\xhat}{\vec{\hat{x}}}
\newcommand{\that}{\hat{t}}

\newcommand{\gammam}{\langle \gamma \rangle}

\newcommand{\Nppc}{N_{\mathrm{ppc}}}

\newcommand{\f}[1]{\ensuremath{f_{\mathrm{ {#1} }}}}
\newcommand{\dt}{\ensuremath{\Delta t}} %delta t
\newcommand{\dd}{\mathrm{d}}


%spesific additions
\newcommand{\runko}{\textsc{Runko}}

\newcommand{\corgi}{\textsc{corgi}}
%\newcommand{\python}{\textsc{Python3}}
\newcommand{\tristan}{\textsc{Tristan}}
\newcommand{\tristanmp}{\textsc{Tristan-MP}}
\newcommand{\pybind}{\textsc{Pybind11}}
\newcommand{\eigen}{\textsc{Eigen}}
\newcommand{\numpy}{\textsc{Numpy}}
\newcommand{\scipy}{\textsc{Scipy}}
\newcommand{\matplotlib}{\textsc{Matplotlib}}


\newcommand{\todo}[1]{\red{#1}}


%________________________________________________________________
\begin{document}


\title{Digital filtering}

\author{J.\,N\"attil\"a\inst{1}}

\institute{Nordita, KTH Royal Institute of Technology and Stockholm University, Roslagstullsbacken 23, SE-10691 Stockholm, Sweden. \email{joonas.nattila@su.se}
}

\date{Received XXX / Accepted XXX}


\abstract{
    Notes about digital (and frequency space) current filtering on particle-in-cell simulations.
}

\keywords{ Plasmas -- Turbulence -- Methods: numerical }


\maketitle

%________________________________________________________________
%\section{Introduction}\label{sec:intro}
%\citep{birdsall1985}

%\section{Theory}\label{sect:theory}

%\section{Results}\label{sect:results}

%\section{Summary}\label{sect:summary}

%\section*{Acknowledgments}
%JN would like to thank XXX stimulating discussions on numerics and plasma simulations. 
%The simulations were performed on resources provided by the Swedish National Infrastructure for Computing (SNIC) at PDC and HPC2N.



%--------------------------------------------------
\section{Current filtering}\label{app:filtering}

See \citet{glasbey1995} for a general discussion about filters.
Here we will mainly focus on linear filters.


%Other citations:
%Shapiro1970

\subsection{Fourier-space filtering}

Convolution of an image $f$ with kernel $w$ (also known as filter mask) to produce a new image $g$ can be expressed as 
\be
g = f \ast w = \int_{\infty}^{\infty} f(\tau) (t-\tau) d\tau. 
\ee
Discrete direct convolution, $g = f \circledast w$, valid for images that are defined on a grid can be expressed as
\be
g_{ij} = \sum_{k=-\frac{n}{2}}^{\frac{n}{2}-1}  \sum_{l=-\frac{n}{2}}^{\frac{n}{2}-1} w_{kl} f_{i+k, j+l},
\ee
for $i,j=1,\ldots,n$ where $(i+k)$ is replaced by $(i+k \pm n)$, if it falls outside the range from $1$ to $n$.
Similar transformation needs to be applied to $(j+l)$.
This is the so-called cyclic (wrapped) direct convolution.
In this case, the design of filters reduces to finding expressions for the kernel $w$.

Fourier transform of $f$, $g$, and $w$ satisfy
\be
g_{kl}^* = w_{kl}^* f_{kl}^*,
\ee
for $k,l = 1,\ldots,n$, provided that $w$ and $f$ are arrays of the same size.
Therefore, filtering in the spatial domain can be expressed with a simple point-by-point multiplication in the frequency domain.
Typically different sources claim that it is beneficial to use Fourier domain for filtering if the kernel is bigger than $7\times7$.
This allows us to design filters in the frequency domains, finding expressions for the $w^*$. 

Ideal low-pass filter is given by setting $w_{kl}^*$ to zero beyond certain distance $R$ from the origin, and otherwise to unity as
\be
w_{kl}^*  =
\begin{cases}
    1 \quad &\text{if } k^2 + l^2 < R^2, \\
    0 \quad &\text{otherwise,} \\
\end{cases}
\ee
for $k,l = -\frac{n}{2},\ldots,\frac{n}{2}-1$.
This will, however, lead to ringing, as is evident from the spatial-domain weights of $w$ that have both negative and positive values.

A filter that minimizes ringing is a Gaussian 
\be
w_{ij} = \frac{1}{\sqrt{2\pi\sigma^2}} \exp \left\{ \frac{-{i^2 + j^2}}{2\sigma^2} \right\},
\ee
where $\sigma$ is the standard deviation or a width of the filter in units of pixels.
Fourier transform of the Gaussian weights can be expressed by
\be
w_{kl}^* = \exp \left\{ \frac{-{k^2 + l^2}}{2[n/(2\pi\sigma)]^2} \right\},
\ee
that is a scaled version of the spatial domain Gaussian, but with a different spread.

In conclusion, a good filter should be resembling a Gaussian as close as possible.

\subsection{Digital filtering}

Digital filtering describes the process of filtering a quantity on the $x$-space (withouth any Fourier transforms to $k$-space).
\red{Hamming1977}
\red{Collatz1966}
\citep{birdsall1985}

Given a grid quantity $X_i \equiv X(i \Delta x)$ a simple digital filtering is performed by replacing
\be
X_i \leftarrow \frac{W X_{i-1} + X_i + W X_{j+1}}{1 + 2W},
\ee
where $W$ is the filter weight.
Caution is needed to make sure that points on the right-hand side are the original values.

Different compact filters can be obtained by selecting different weightings.
A filter with $W=\frac{1}{2}$ is called a binomial filter and has a positive smoothing function in $k$-space.
Selecting $W < 0$ produces a so-called compensation filter as it enhances the original signal.
$W = -\frac{1}{6}$ gives a compensation that cancels second-order attenuation of a $W=\frac{1}{2}$ filter, i.e., applying a $W=\frac{1}{2}$ filter followed by a $W=-\frac{1}{6}$ produces a fourth-order attenuated filter corresponding to a broader stencil with weights of $(1/16)[-1\quad 4\quad 10\quad 4\quad -1]$.

All of the filtering operations above need a temporary scratch array of the quantity being processed.
A scratch-memory-free alternatives can be derived by assuming multiple passes with different filters.
Simplest one can be obtained by replacing $X_i$ with a forward-pass of $X_i + U X_{i+1}$ and a backward-pass of $X_i + U X_{i-1}$, for a multi-pass filter parameter $U$.
This is equal to the one-pass filter for
\be
W = \frac{U}{1+U^2}
\ee
or
\be
U = \frac{1}{2W} \pm \sqrt{ \frac{1}{(2W)^2} - 1}.
\ee
For $U$ to be real valued, we require $-\frac{1}{2} \leq W \leq \frac{1}{2}$.
Binomial two-pass filter (equal to $\frac{1}{4}[1\quad 2\quad 1]$; $W=\frac{1}{2}$) corresponds to $U=1$ with a forward-pass of $\frac{1}{2}[0\quad1\quad1]$ and a return-pass with $\frac{1}{2}[1\quad1\quad0]$.
The corresponding two-pass compensator ($W=-\frac{1}{6}$) has $U=-3+\sqrt{8} \approx -0.171573$ to be applied after the smoothing.

Two-dimensional filters can be generated by using two one-dimensional filters (shown here for the second-order Binomial filter $B_2$) as 
\be
B_2^{(2D)} \equiv B_2 \otimes B_2^T = 
\frac{1}{4} [1\quad 2\quad 1] \otimes 
\frac{1}{4}
\begin{bmatrix}
    1 \\
    2 \\
    1 \\
\end{bmatrix}
= \frac{1}{16}
\begin{bmatrix}
    1 & 2 & 1 \\
    2 & 4 & 1 \\
    1 & 2 & 1 \\
\end{bmatrix}.
\ee
A scratch-memory-free version also exists in a form of a four-pass filter (consisting of $C_{\rightarrow} \equiv \frac{1}{2}[0\quad 1\quad 1]$ and $C_{\leftarrow} \equiv \frac{1}{2}[1\quad 1\quad 0]$) as
\begin{align}
B_2^{(2D)} 
    =& C_{\rightarrow} \otimes C_{\leftarrow} \otimes C_{\rightarrow}^T \otimes C_{\leftarrow}^T \\
    =& 
\frac{1}{2}
\begin{bmatrix}
    0 & 0 & 0 \\
    0 & 1 & 1 \\
    0 & 0 & 0 \\
\end{bmatrix}
\frac{1}{2}
\begin{bmatrix}
    0 & 0 & 0 \\
    1 & 1 & 0 \\
    0 & 0 & 0 \\
\end{bmatrix}
\frac{1}{2}
\begin{bmatrix}
    0 & 1 & 0 \\
    0 & 1 & 0 \\
    0 & 0 & 0 \\
\end{bmatrix}
\frac{1}{2}
\begin{bmatrix}
    0 & 0 & 0 \\
    0 & 1 & 0 \\
    0 & 1 & 0 
\end{bmatrix}.
\end{align}
A possible choice for the compensator can be $(20,-1,-1)$ or $(36, -6, 1)$
\red{check Birdsall filter notation assumed to be $(\mathrm{center}, \mathrm{middle}, \mathrm{corner})$} 
\be
\begin{bmatrix}
    -1 & -1 & -1 \\
    -1 & 20 & -1 \\
    -1 & -1 & -1 
\end{bmatrix}
\quad
\mathrm{or}
\quad
\begin{bmatrix}
     1 & -6 &  1 \\
    -6 & 36 & -6 \\
     1 & -6 &  1 
\end{bmatrix}.
\ee
Here, one should, however, note that scratch-free versions can not be vectorized, which means that for modern processing units are probably not the optimal choice.

From the numerical point-of-view the filtering is always a balance between number of passes and size of the filter stencil.
It also matters in what order we are asking the variables from the array; a modern processor has a prefetcher that can try to guess our memory access patterns. 
A suitable selection for the $2D$ seems to be a two-pass filtering (up and down) with the compact $3$-point binomial (requiring $3$ values, $X_{i-1}, X_i$, and $X_{i+1}$).
A modern prefetcher might even be able to cope with a one-pass compact Binomial (requiring $9$ values in $2D$).

Note also that thanks to the usage of tiles (with a lenght of $3$ buffer zones for the current arrays), it can make sense to combine multiple compact Binomial passes into a fewer passes with an extended stencil (supporting a $6$ order stencil with $7$ [$3+1+3$] points with one MPI communication round).


\subsection{Binomial filters}

Binomial filters form a compact approximation for the (discretized) Gaussian.\footnote{\url{http://www.cse.yorku.ca/~kosta/CompVis_Notes/binomial_filters.pdf}} 
    \red{Hamming 1977}
Binomial coefficients are given by
\be
\begin{pmatrix}
    N \\
    n \\
\end{pmatrix}
= \frac{N!}{(N-n)! N!},
\ee
where $n=0,\ldots,N$ and $N$ is the order of the binomial filter $B_N$.
It forms a rapid approximation for the Gaussian distribution as $\sigma \approx \sqrt{N}/2$.
\footnote{Alternatively, binomial filters correspond to the rows of the Pascal's triangle.}
First few are given as
\begin{align}
B_0 &=  [1] \\
B_1 &= \frac{1}{2} [1\quad 1] \\
B_2 &= \frac{1}{4} [1\quad 2\quad 1] \\
B_3 &= \frac{1}{8} [1\quad 3\quad 3\quad 1] \\
B_4 &= \frac{1}{16} [1\quad 4\quad 6\quad 4\quad 1] \\
B_5 &= \frac{1}{32} [1\quad 5\quad 10\quad 10\quad 5\quad 1] \\
B_6 &= \frac{1}{64} [1\quad 6\quad 15\quad 20\quad 15\quad 6\quad 1] \\
\vdots
\end{align}
The binomial filter coefficients are applied for a quantity $X$ (defined on a grid as $X_i \equiv X(i\Delta x)$ as
\be
X_i^{f_2} = B_2\{ X_i \} = 
    \frac{1}{4} X_{i-1} + 
    \frac{1}{2} X_i  + 
    \frac{1}{4} X_{i+1},
\ee
where $X_i^{f_2}$ the filtered quantity and $B_2\{X\}$ denotes an application of a $B_2$ filter on the original quantity.
Applying higher-order binomial filters generalizes naturally from here.

Discrete-time Fourier transform of the binomial filter $B_2 = \frac{1}{4}[1 \quad 2 \quad 1]$ is proportional to a single period of a cosine as 
\be
B_2^*[k] = \sum_n B_2[n] \cos[k n] = \frac{1}{2} + \frac{1}{2} \cos[k].
\ee
For higher-order filters of rank $N$ we similarly have
\be
B_N^*[k] = \left( \frac{1}{2} + \frac{1}{2}\cos(k) \right)^N.
\ee



%Tristan has 
%binomial filters
%optimized filters (x,y,z directions; y and z avoid cross-memory accesses)
%chunk'd filters


\subsection{Temporal filters}
Another possibility, besides usage of spatial filters, is to filter electric field in time.
This is known as the Friedman filter, defined as
\be
\vec{E}^{(n)} = 
    \left(1+\frac{\theta}{2}\right) \vec{E}^{(n)}
   -\left(1-\frac{\theta}{2}\right) \vec{E}^{(n-1)}
   +\frac{1}{2} (1-\theta)^2 \bar{\vec{E}}^{(n-2)},
\ee
where $\bar{\vec{E}}^{(n-1)} = \vec{E}^{(n-2)} + \theta \vec{E}^{(n-3)}$, and the filtering parameter $\theta \in [0,1]$.


\bibliographystyle{aa}
\bibliography{refs}

%\begin{thebibliography}{62}
%\end{thebibliography}

%\clearpage
\onecolumn
\begin{appendix}
\end{appendix}

\end{document}
