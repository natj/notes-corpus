%\documentclass[referee]{aa}
\documentclass{aa}


\usepackage[utf8]{inputenc}
\usepackage{tabularx}
\usepackage{multirow}
\usepackage{url}
\usepackage{amsmath}	% Advanced maths commands
\usepackage{amssymb}	% Extra maths symbols
\usepackage{graphicx}
\usepackage{color}
\usepackage{natbib}
\usepackage{hyperref}

%\usepackage{newtxtext,newtxmath}
\usepackage{times,txfonts}
\usepackage{bm}

% Depending on your LaTeX fonts installation, you might get better results with one of these:
%\usepackage{mathptmx}
%\usepackage{txfonts}

% Use vector fonts, so it zooms properly in on-screen viewing software
% Don't change these lines unless you know what you are doing
\usepackage[T1]{fontenc}
\usepackage{ae,aecompl}


%landscape figures
%\usepackage{wrapfig}
%\usepackage{lscape}
%\usepackage{rotating}

\bibpunct{(}{)}{;}{a}{}{,} % to follow the A&A style

%Debug addition for collaborators
%\usepackage[switch, modulo]{lineno}
%\linenumbers
%\renewcommand\linenumberfont{\color{red}\normalfont\tiny\sffamily}
%\renewcommand\linenumberfont{\normalfont\tiny\sffamily}

\DeclareUnicodeCharacter{00A0}{ }

\usepackage{listingsutf8}

% configure listings package
\lstset{
         basicstyle=\footnotesize\ttfamily, 
         %numbers=left,               
         numberstyle=\tiny,          
         %stepnumber=2,             
         numbersep=5pt,             
         tabsize=2,                  
         extendedchars=true,         
         breaklines=true,            
         keywordstyle=\color{red},
         frame=b,
 %        keywordstyle=[1]\textbf,   
 %        keywordstyle=[2]\textbf,   
 %        keywordstyle=[3]\textbf,   
 %        keywordstyle=[4]\textbf,   
         %stringstyle=\color{white}\ttfamily, % Farbe der String
         showspaces=false,           
         showtabs=false,            
         xleftmargin=17pt,
         framexleftmargin=17pt,
         framexrightmargin=5pt,
         framexbottommargin=4pt,
         %backgroundcolor=\color{lightgray},
         showstringspaces=false    
}
\lstloadlanguages{C++}
\newcommand{\codeil}[1]{\lstinline!#1!}


\usepackage{amsmath}
\usepackage{amssymb}


%--------------------------------------------------
%Basic commands
\newcommand{\be}{\begin{equation}}
\newcommand{\ee}{\end{equation}}
\newcommand{\bea}{\begin{align}\begin{split}}
\newcommand{\eea}{\end{split}\end{align}}

\newcommand{\fig}[1]{Fig.~\ref{#1}}
\newcommand{\eq}[1]{~(\ref{#1})}
\newcommand{\sect}[1]{Sect.~\ref{#1}}
\newcommand{\appnd}[1]{Appendix \ref{#1}}

\newcommand{\Ten}[2]{\ensuremath{#1 \times 10^{#2}} }


%-------------------------------------------------- 
% Units with proper unit spacing
\newcommand{\unitspace}{\,}

\newcommand{\km}{\ensuremath{\unitspace \mathrm{km}}}
\newcommand{\cm}{\ensuremath{\unitspace \mathrm{cm}}}
\newcommand{\g}{\ensuremath{\unitspace \mathrm{g}}}
\newcommand{\Hz}{\ensuremath{\unitspace \mathrm{Hz}}}
\newcommand{\gcm}{\ensuremath{\unitspace \mathrm{g}\,\mathrm{cm}^{-3}}}
\newcommand{\cms}{\ensuremath{\unitspace \mathrm{cm}\,\mathrm{s}^{-1}}}
\newcommand{\cmss}{\ensuremath{\unitspace \mathrm{cm}\,\mathrm{s}^{-2}}}
\newcommand{\dyncm}{\ensuremath{\unitspace \mathrm{dyn}\,\mathrm{cm}^{-2}}}
\newcommand{\erg}{\ensuremath{\unitspace \mathrm{erg}}}
\newcommand{\ergs}{\ensuremath{\unitspace \mathrm{erg}\,\mathrm{s}^{-1}}}

\newcommand{\Gauss}{\ensuremath{\unitspace \mathrm{G}}}
\newcommand{\Kelvin}{\ensuremath{\unitspace \mathrm{K}}}

%\newcommand{\Msun}{\ensuremath{\unitspace M_{\odot}}}
\newcommand{\Msun}{\ensuremath{\unitspace \mathrm{M}_{\odot}}}


%--------------------------------------------------
% colors
\newcommand{\red}[1]{\textcolor{red}{#1}}
\newcommand{\green}[1]{\textcolor{green}{#1}}
\newcommand{\blue}[1]{\textcolor{blue}{#1}}



%--------------------------------------------------
% vectors

%normal 3-vectors
%\renewcommand{\vec}[1]{\ensuremath{\mathbf{#1}}}
%\renewcommand{\vec}[1]{\ensuremath{\boldsymbol{#1}}}
\renewcommand{\vec}[1]{\ensuremath{\bm{\mathit{#1}}}}

%four-vectors
\makeatletter
\def\fvec#1{\underline{\sbox\tw@{$#1$}\dp\tw@\z@\box\tw@}}
\makeatother


%--------------------------------------------------
% math

% integrals
\newcommand{\ud}{\mathrm{d}} % differential d

% derivatives
\newcommand{\pD}[2]{\ensuremath{\partial_{#1}}} %partial derivative
\newcommand{\fracpD}[2]{\ensuremath{\frac{\partial {#1} }{\partial {#2} }}} %frac partial

\newcommand{\derfrac}[2]{\frac{\ud #1}{\ud #2}} % fractional deriv
\newcommand{\parfrac}[2]{\frac{\partial #1}{\partial #2}} %fractional partial deriv
\newcommand{\parD}[1]{\partial_{#1}} %partial derivative shorthand

%misc
\newcommand{\osim}{\mathord{\sim}} % ordo O

% statistics
\newcommand{\bra}[1]{\ensuremath{\langle #1 \rangle}}
\newcommand{\braket}[2]{\ensuremath{ \langle #1 ~|~ #2 \rangle}}


%--------------------------------------------------
% programming

% writing c++ is highly non-trivial
%\newcommand{\cpp}{C\nolinebreak\hspace{-.05em}\raisebox{.4ex}{\tiny\bf +}\nolinebreak\hspace{-.10em}\raisebox{.4ex}{\tiny\bf +{rbrace}{rbrace}}}
%\newcommand{\cpp}{C\nolinebreak\hspace{-.05em}\raisebox{.4ex}{\tiny\bf +}\nolinebreak\hspace{-.10em}\raisebox{.4ex}{\tiny\bf +}}
%\def\cpp{{C\nolinebreak[4]\hspace{-.05em}\raisebox{.4ex}{\tiny\bf ++}}}
%\def\CC{lbrace}{lbrace}C\nolinebreak[4]\hspace{-.05em}\raisebox{.4ex}{\tiny\bf ++{rbrace}{rbrace}{rbrace}
%\usepackage{relsize}
%\newcommand\cpp{C\nolinebreak\hspace{-.05em}\raisebox{.4ex}{\relsize{-3}{\textbf{+}}}\nolinebreak\hspace{-.10em}\raisebox{.4ex}{\relsize{-3}{\textbf{+}}}}
%\newcommand\cpp{C\nolinebreak[4]\hspace{-.05em}\raisebox{.4ex}{\relsize{-3}{\textbf{++}}}}

\newcommand{\cpp}{C\texttt{++}}
\newcommand{\cppv}[1]{C\texttt{++}$#1$}

\newcommand{\python}{\textsc{Python}}





\newcommand{\Cfl}{\hat{c}}
\newcommand{\hf}{\frac{1}{2}}

\newcommand{\Bhat}{\vec{\hat{B}}}
\newcommand{\Ehat}{\vec{\hat{E}}}
\newcommand{\Jhat}{\vec{\hat{J}}}
\newcommand{\uhat}{\vec{\hat{u}}}
\newcommand{\vhat}{\vec{\hat{v}}}
\newcommand{\xhat}{\vec{\hat{x}}}
\newcommand{\that}{\hat{t}}

\newcommand{\gammam}{\langle \gamma \rangle}

\newcommand{\Nppc}{N_{\mathrm{ppc}}}

\newcommand{\f}[1]{\ensuremath{f_{\mathrm{ {#1} }}}}
\newcommand{\dt}{\ensuremath{\Delta t}} %delta t
\newcommand{\dd}{\mathrm{d}}


%spesific additions
\newcommand{\runko}{\textsc{Runko}}

\newcommand{\corgi}{\textsc{corgi}}
%\newcommand{\python}{\textsc{Python3}}
\newcommand{\tristan}{\textsc{Tristan}}
\newcommand{\tristanmp}{\textsc{Tristan-MP}}
\newcommand{\pybind}{\textsc{Pybind11}}
\newcommand{\eigen}{\textsc{Eigen}}
\newcommand{\numpy}{\textsc{Numpy}}
\newcommand{\scipy}{\textsc{Scipy}}
\newcommand{\matplotlib}{\textsc{Matplotlib}}


\newcommand{\todo}[1]{\red{#1}}


%________________________________________________________________
\begin{document}


\title{Relativistic plasma physics}

\author{J.\,N\"attil\"a\inst{1}}

\institute{Nordita, KTH Royal Institute of Technology and Stockholm University, Roslagstullsbacken 23, SE-10691 Stockholm, Sweden. \email{joonas.nattila@su.se}
}

\date{Received XXX / Accepted XXX}


\abstract{
    Basic plasma parameters are presented and succintly discussed here.
}

\keywords{ Plasmas -- Turbulence -- Methods: numerical }


\maketitle

%________________________________________________________________
%\section{Introduction}\label{sec:intro}

\section{Theory}\label{sect:theory}

\subsection{Plasma parameters}

Two important frequencies in a plasma are the plasma frequency,
\be
\omega_p \equiv \sqrt{\frac{4\pi n q^2 }{\gamma_0 m_e}},
\ee
and cyclotron frequency,
\be
\omega_B \equiv \frac{q B}{\gamma_0 m_e c}.
\ee
Particle Lorentz factor is defined as
\be
\gamma = \sqrt{ \frac{1}{1-\beta^2}} = \sqrt{1 + \frac{u^2}{c^2}},
\ee
where $\beta$ is the particle coordinate velocity and $u$ is the (spatial component of) four-velocity.
Additionally, $u/c = \gamma \beta$.
The relativistic Lorentz factor, $\gamma_0$, present in above formulae can either be some bulk Lorentz factor of the flow ($\gamma_0 = \Gamma$) or the mean thermal Lorentz factor of the plasma ($\gamma_0 = \langle \gamma \rangle$), depending on the problem at hand.

The frequencies can be used to define two characteristic length scales, skin depth
\be
\lambda_p \equiv \frac{c}{\omega_p},
\ee
and Larmor radius (gyrorotation orbit)
\be
\lambda_L \equiv \frac{c}{\omega_B}.
\ee

Additionally, we note that the magnetization, defined more rigorously later on, is related to the above frequencies and length scales as
\be
\sigma = \frac{\omega_B^2}{\omega_p^2} = \frac{\lambda_p^2}{\lambda_L^2}.
\ee

Note also that relativistic time dilatation and length stretching work such that
$\omega_{p,\mathrm{rel}} = \omega_p / \sqrt{\gamma_0}$,
$\lambda_{p,\mathrm{rel}} = \lambda_p \sqrt{\gamma_0}$, and
$\omega_{B,\mathrm{rel}} = \omega_B / \gamma_0$,
where $X_{\mathrm{rel}}$ are the ``true'' relativistic counterparts to the classical plasma values.



\subsection{Hot plasmas}

Plasma temperature in units of electron rest mass is
\be
\theta = \frac{kT}{m_e c^2}.
\ee
Plasma is considered hot when $\theta \gtrsim 1$.

Particle distribution of a relativistic plasma in thermodynamic equilibrium follows Maxwell-J\"uttner distribution,
\be
M(\vec{u}) = \frac{1}{4\pi K_2(\theta^{-1})} \exp \left( -\frac{\sqrt{1+\gamma^2 \beta^2}}{\theta} \right),
\ee
where $K_2$ is the modified Bessel function of the second kind.
In terms of Lorentz factor it is
\be
M(\gamma) = \frac{\gamma^2 \beta }{K_2(\theta^{-1}) \theta} \exp\left( -\frac{\gamma}{\theta} \right).
\ee

Maxwell-J\"uttner distribution has a few useful averages that have a physical correspondence.
First, we note that 
\be
\kappa_{ij}(x) = \frac{K_i(x)}{K_j(x)} = 
\begin{cases}
    1 + \frac{5}{2} \theta, & \text{for } \theta \rightarrow 0      \text{ (non-relativistic)}\\
    4 \theta,               & \text{for } \theta \rightarrow \infty \text{ (ultra-relativistic)}\\
\end{cases},
\ee
where $K_n$ is the modified Bessel function of the $n$th kind.

A mean Lorentz factor is
\be
    %\langle \gamma \rangle = \frac{K_3(1/\theta)}{K_2(1/\theta)} - \frac{1}{\theta}.
    \langle \gamma \rangle = \kappa_{32}(\theta^{-1}) - \theta^{-1} = 
\begin{cases}
    1 + \frac{3}{2} \theta, & \text{for } \theta \rightarrow 0      \text{ (non-relativistic)}\\
    3 \theta,               & \text{for } \theta \rightarrow \infty \text{ (ultra-relativistic)}\\
\end{cases}.
\ee
Note that $\langle \gamma \rangle \approx 1 + 3\theta$, which approximatively captures both limits and therefore mimicks the thermal Lorentz factor of a plasma particle.
%In the cold non-relativistic limit, $\theta \rightarrow 0$, this approaches $\langle \gamma \rangle \rightarrow 1 + \frac{3}{2}\theta$ and in the hot ultra-relativistic limit, $\theta \rightarrow \infty$, $\langle \gamma \rangle \rightarrow 3\theta$.
Similarly, 
\be
    \langle \gamma^2 \beta^2 \rangle = 3 \kappa_{32}(\theta^{-1}) = 
\begin{cases}
    3\theta,                   & \text{for } \theta \rightarrow 0      \text{ (non-relativistic)}\\
    12 \theta^2,               & \text{for } \theta \rightarrow \infty \text{ (ultra-relativistic)}\\
\end{cases}.
\ee
%that in the cold limit $\rightarrow 3\theta$ and hot limit $\rightarrow 12\theta^2$.

Average Larmor radius of a thermal plasma is
\be
%\lambda_L' = \lambda_L \sqrt{2\theta \frac{K_3(1/\theta)}{K_2(1/\theta)} }.
\lambda_L' = \lambda_L \sqrt{2\theta \kappa_{32}(\theta^{-1})}
\ee
Thermal pressure is
\be
P = \frac{1}{3} n \langle \vec{v} \cdot \gamma m \vec{v} \rangle = n \theta m_e c^2.
\ee
Enthalpy is
\begin{align}
%h = \frac{n \langle \gamma m c^2 \rangle + P }{n m c^2} = \frac{K_3(1/\theta)}{K_2(1/\theta)},
    h &= \frac{n \langle \gamma m c^2 \rangle + P }{n m c^2} = \kappa_{32}(\theta^{-1}) \\
    &= 
\begin{cases}
    1 + \frac{5}{2} \theta, & \text{for } \theta \rightarrow 0      \text{ (non-relativistic)}\\
    4 \theta,               & \text{for } \theta \rightarrow \infty \text{ (ultra-relativistic)}\\
\end{cases}.
\end{align}
%that in the cold limit $\rightarrow 1 + \frac{5}{2} \theta$ and in the hot limit $\rightarrow 4 \theta$. 
Adiabatic exponent,
\begin{align}
    \Gamma_{\mathrm{ad}} &= 1 + \frac{P}{n \langle \gamma - 1 \rangle m c^2} 
    %= 1 + \left( \frac{1}{\theta} \frac{K_3(1/\theta)}{K_2(1/\theta)} - \frac{1}{\theta} - 1 \right)^{-1},
    = 1 + \left( \frac{1}{\theta} \kappa_{32}(\theta^{-1}) - \theta^{-1} - 1 \right)^{-1} \\
    &= 
\begin{cases}
    \frac{5}{3}, & \text{for } \theta \rightarrow 0      \text{ (non-relativistic)}\\
    \frac{4}{3},               & \text{for } \theta \rightarrow \infty \text{ (ultra-relativistic)}\\
\end{cases}.
\end{align}
%that in the cold limit $\rightarrow \frac{5}{3}$ and in the hot limit $\rightarrow \frac{4}{3}$.


A plasma moving with a bulk drift coordinate-velocity $\vec{V}_0$ (with Lorentz factor $\Gamma_0$) to the $y$ direction ($\vec{V}_0 = V_0 \hat{\vec{y}}$), on the other hand, has
\be
M(\vec{u})' = \frac{1}{4\pi K_2(\theta^{-1}) \Gamma_0} \exp \left( -\Gamma_0 \frac{-V_0^2 \Gamma_0 + \sqrt{1+\gamma^2 \beta^2}}{\theta} \right),
\ee
and number density in that frame is $n = \Gamma_0 n_0$.
Mean four-velocity in this case is
\begin{align}
    \langle \gamma \vec{v} \rangle &= 
        \kappa_{32}(\theta^{-1}) \Gamma_0 \vec{V}_0  \\
    &= 
\begin{cases}
    \Gamma_0 \vec{V}_0,          & \text{for } \theta \rightarrow 0      \text{ (non-relativistic)}\\
    4\theta \Gamma_0 \vec{V}_0,  & \text{for } \theta \rightarrow \infty \text{ (ultra-relativistic)}\\
\end{cases}
\end{align}
and mean Lorentz factor is
\begin{align}
    \langle \gamma \rangle &= 
        \kappa_{32}(\theta^{-1}) \Gamma_0 - (\Gamma_0 \theta )^{-1} \\
    &= 
\begin{cases}
    \Gamma_0 ,          & \text{for } \theta \rightarrow 0      \text{ (non-relativistic)}\\
    4\theta \Gamma_0 - (\Gamma_0 \theta)^{-1},  & \text{for } \theta \rightarrow \infty \text{ (ultra-relativistic)}\\
\end{cases}
\end{align}.
Note that the perpendicular to the drift direction, the mean four-velocity fluxes (related to pressure) are
\be
    \langle u_x v_x \rangle = 
    \langle u_z v_z \rangle = 
    \theta/\Gamma_0,
\ee
whereas to the drift direction we have
\be
    \langle u_y v_y \rangle = 
    \theta/\Gamma_0 + \Gamma_0 V_0^2 \kappa_{32}(\theta^{-1}).
\ee




\subsection{Magnetized plasmas}
Magnetic field energy density is
\be
U_B \equiv \frac{B^2}{8\pi}
\ee
and electric field energy density is
\be
U_E \equiv \frac{E^2}{8\pi}.
\ee
Similarly, internal energy content density of the plasma can be quantified with the enthalpy density
\be
w \equiv \gamma n_e m_e c^2.
\ee

For magnetized plasmas it is useful to define the magnetization parameter that is the ratio of the magnetic energy density to the plasma enthalpy density.
For a cold, non-relativistic plasma it reduces to 
\be
\sigma_c = \frac{B^2}{4\pi n m_e c^2}.
\ee
A hot plasma ($\theta \equiv kT/m_e c^2 \gtrsim 1$) carries a relativistic energy content and so the true ``hot'' magnetization is
\be
\sigma = \frac{B^2}{4\pi n \gamma_{\mathrm{th}} m_e c^2} 
        \approx \frac{B^2}{4\pi n (1+3\theta) m_e c^2},
\ee
where $\gamma_{\mathrm{th}} = \langle \gamma \rangle$ is the average Lorentz factor of a hot plasma.
Similarly, a plasma with a considerable relativistic bulk motion ($\Gamma > 1$) has a true ``relativistic'' magnetization of
\be
\sigma = \frac{B^2}{4\pi n \Gamma m_e c^2}.
\ee


\section{Plasma instabilities}

\red{TODO:}

\subsection{Plasma dispersion relations}

\subsection{Two-stream instability}

\subsection{Current filamentation instability}

\subsection{Weibel instability}

\subsection{Tearing mode instability}

Tearing mode dispersion relation from the relativistic pair-plasma fluid equations (e.g., Koide 2009)\\
by applying the standard tearing mode analysis\\
Furth 1963\\
Coppi 1976\\
Ara 1978\\
non-relativistic one Porcelli 1991\\


%\section{Results}\label{sect:results}

\section{Literature}\label{sect:summary}

On PIC: \citet{birdsall1985}.
On useful plasma parameters: \citet{Melzani_2014}


%\section*{Acknowledgments}
%JN would like to thank XXX stimulating discussions on numerics and plasma simulations. 
%The simulations were performed on resources provided by the Swedish National Infrastructure for Computing (SNIC) at PDC and HPC2N.


\bibliographystyle{aa}
\bibliography{refs}

%\begin{thebibliography}{62}
%\end{thebibliography}

\clearpage
%\onecolumn
\begin{appendix}

\section{\textsc{Runko} code units}

%Useful facts about code units: Electron plasma frequency in code units (inverse timesteps) is omega_pe = sqrt( q_e n / ((m/m_e) gamma)), where q is the charge of the particle, n density and gamma is the typical lorenz factor, m is particle’s mass. Note that here 4 pi = 1 compared to cgs, and q_e/m_e = 1 by definition. Cyclotron frequency: omega_c = B/(gamma (m/m_e) c) (in inverse time steps). Magnetization sigma = (omega_c)^2/ (omega_p)^2 (c/v)^2 = 1/M_A^2, where M_A is the alfvenic Mach #.

Time in \textsc{Runko} code units is related to the plasma frequency as
\be
\hat{\omega}_{p} \Delta t = \sqrt{\frac{ \hat{n} \hat{q} }{ \gamma_0 \hat{m} } },
\ee
where $\hat{q}$ is the particle charge normalization (\texttt{qe} or \texttt{qi}), $\hat{n}$ is the (numerical) total particle number density (typically $2\times$\texttt{ppc}), and $\hat{m} = m/m_e$ is the  mass in units of electron masses.
Note that here $4\pi = 1$ and $q_e/m_e = 1$ by definition.
Cyclotron frequency is similarly 
\be
\hat{\omega}_B \Delta t = \frac{\hat{B}}{\gamma_0 \hat{m} \hat{C}},
\ee
    where $\hat{B}$ is the magnetic field in code units and $\hat{C}$ is the numerical speed of light in the simulation (CFL number; \texttt{cfl}).
Magnetization is
\be
\sigma = \frac{\hat{\omega}_B^2}{\hat{\omega}_p^2}.
\ee

Typical number density per ``pixel'' is
\be
    \hat{n}_0 = 2 n_{\mathrm{ppc}} \mathcal{S}^2,
\ee
where $n_{\mathrm{ppc}}$ is the particle per cell per species (\texttt{ppc}), and $\mathcal{S}$ is the output striding factor (\texttt{stride}).
A standard charge density normalization is then
\be
    \hat{\rho}_0 = \hat{q}_e \hat{n}_0.
\ee

Similarly, the current density can be normalized with
\be
    \hat{J}_0 = \hat{q}_e \hat{n}_0 \hat{\mathcal{C}}^2,
\ee
Resulting in current density units of $q_e n_0 c$.
Note that the extra $\hat{\mathcal{C}}$ factor in the above equation originates from the fact that the numerical current stored in the memory is actually $\hat{J}\Delta t$.

Fields can be normalized with
\be
    \hat{B}_0 = \frac{\hat{m}}{\hat{q}_e} \hat{\mathcal{C}}^2 \mathcal{R}_{\lambda},
\ee
where grid resolution in units of skin depth is $\mathcal{R}_{\lambda} = \lambda_p/\Delta x$ (\texttt{c\_omp}).
This gives fields units of $\omega_p m_e / q_e$.


\end{appendix}


\end{document}
